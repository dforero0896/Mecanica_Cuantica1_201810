
% Default to the notebook output style

    


% Inherit from the specified cell style.




    
\documentclass[11pt]{article}

    
    
    \usepackage[T1]{fontenc}
    % Nicer default font (+ math font) than Computer Modern for most use cases
    \usepackage{mathpazo}

    % Basic figure setup, for now with no caption control since it's done
    % automatically by Pandoc (which extracts ![](path) syntax from Markdown).
    \usepackage{graphicx}
    % We will generate all images so they have a width \maxwidth. This means
    % that they will get their normal width if they fit onto the page, but
    % are scaled down if they would overflow the margins.
    \makeatletter
    \def\maxwidth{\ifdim\Gin@nat@width>\linewidth\linewidth
    \else\Gin@nat@width\fi}
    \makeatother
    \let\Oldincludegraphics\includegraphics
    % Set max figure width to be 80% of text width, for now hardcoded.
    \renewcommand{\includegraphics}[1]{\Oldincludegraphics[width=.8\maxwidth]{#1}}
    % Ensure that by default, figures have no caption (until we provide a
    % proper Figure object with a Caption API and a way to capture that
    % in the conversion process - todo).
    \usepackage{caption}
    \DeclareCaptionLabelFormat{nolabel}{}
    \captionsetup{labelformat=nolabel}

    \usepackage{adjustbox} % Used to constrain images to a maximum size 
    \usepackage{xcolor} % Allow colors to be defined
    \usepackage{enumerate} % Needed for markdown enumerations to work
    \usepackage{geometry} % Used to adjust the document margins
    \usepackage{amsmath} % Equations
    \usepackage{amssymb} % Equations
    \usepackage{textcomp} % defines textquotesingle
    % Hack from http://tex.stackexchange.com/a/47451/13684:
    \AtBeginDocument{%
        \def\PYZsq{\textquotesingle}% Upright quotes in Pygmentized code
    }
    \usepackage{upquote} % Upright quotes for verbatim code
    \usepackage{eurosym} % defines \euro
    \usepackage[mathletters]{ucs} % Extended unicode (utf-8) support
    \usepackage[utf8x]{inputenc} % Allow utf-8 characters in the tex document
    \usepackage{fancyvrb} % verbatim replacement that allows latex
    \usepackage{grffile} % extends the file name processing of package graphics 
                         % to support a larger range 
    % The hyperref package gives us a pdf with properly built
    % internal navigation ('pdf bookmarks' for the table of contents,
    % internal cross-reference links, web links for URLs, etc.)
    \usepackage{hyperref}
    \usepackage{longtable} % longtable support required by pandoc >1.10
    \usepackage{booktabs}  % table support for pandoc > 1.12.2
    \usepackage[inline]{enumitem} % IRkernel/repr support (it uses the enumerate* environment)
    \usepackage[normalem]{ulem} % ulem is needed to support strikethroughs (\sout)
                                % normalem makes italics be italics, not underlines
    

    
    
    % Colors for the hyperref package
    \definecolor{urlcolor}{rgb}{0,.145,.698}
    \definecolor{linkcolor}{rgb}{.71,0.21,0.01}
    \definecolor{citecolor}{rgb}{.12,.54,.11}

    % ANSI colors
    \definecolor{ansi-black}{HTML}{3E424D}
    \definecolor{ansi-black-intense}{HTML}{282C36}
    \definecolor{ansi-red}{HTML}{E75C58}
    \definecolor{ansi-red-intense}{HTML}{B22B31}
    \definecolor{ansi-green}{HTML}{00A250}
    \definecolor{ansi-green-intense}{HTML}{007427}
    \definecolor{ansi-yellow}{HTML}{DDB62B}
    \definecolor{ansi-yellow-intense}{HTML}{B27D12}
    \definecolor{ansi-blue}{HTML}{208FFB}
    \definecolor{ansi-blue-intense}{HTML}{0065CA}
    \definecolor{ansi-magenta}{HTML}{D160C4}
    \definecolor{ansi-magenta-intense}{HTML}{A03196}
    \definecolor{ansi-cyan}{HTML}{60C6C8}
    \definecolor{ansi-cyan-intense}{HTML}{258F8F}
    \definecolor{ansi-white}{HTML}{C5C1B4}
    \definecolor{ansi-white-intense}{HTML}{A1A6B2}

    % commands and environments needed by pandoc snippets
    % extracted from the output of `pandoc -s`
    \providecommand{\tightlist}{%
      \setlength{\itemsep}{0pt}\setlength{\parskip}{0pt}}
    \DefineVerbatimEnvironment{Highlighting}{Verbatim}{commandchars=\\\{\}}
    % Add ',fontsize=\small' for more characters per line
    \newenvironment{Shaded}{}{}
    \newcommand{\KeywordTok}[1]{\textcolor[rgb]{0.00,0.44,0.13}{\textbf{{#1}}}}
    \newcommand{\DataTypeTok}[1]{\textcolor[rgb]{0.56,0.13,0.00}{{#1}}}
    \newcommand{\DecValTok}[1]{\textcolor[rgb]{0.25,0.63,0.44}{{#1}}}
    \newcommand{\BaseNTok}[1]{\textcolor[rgb]{0.25,0.63,0.44}{{#1}}}
    \newcommand{\FloatTok}[1]{\textcolor[rgb]{0.25,0.63,0.44}{{#1}}}
    \newcommand{\CharTok}[1]{\textcolor[rgb]{0.25,0.44,0.63}{{#1}}}
    \newcommand{\StringTok}[1]{\textcolor[rgb]{0.25,0.44,0.63}{{#1}}}
    \newcommand{\CommentTok}[1]{\textcolor[rgb]{0.38,0.63,0.69}{\textit{{#1}}}}
    \newcommand{\OtherTok}[1]{\textcolor[rgb]{0.00,0.44,0.13}{{#1}}}
    \newcommand{\AlertTok}[1]{\textcolor[rgb]{1.00,0.00,0.00}{\textbf{{#1}}}}
    \newcommand{\FunctionTok}[1]{\textcolor[rgb]{0.02,0.16,0.49}{{#1}}}
    \newcommand{\RegionMarkerTok}[1]{{#1}}
    \newcommand{\ErrorTok}[1]{\textcolor[rgb]{1.00,0.00,0.00}{\textbf{{#1}}}}
    \newcommand{\NormalTok}[1]{{#1}}
    
    % Additional commands for more recent versions of Pandoc
    \newcommand{\ConstantTok}[1]{\textcolor[rgb]{0.53,0.00,0.00}{{#1}}}
    \newcommand{\SpecialCharTok}[1]{\textcolor[rgb]{0.25,0.44,0.63}{{#1}}}
    \newcommand{\VerbatimStringTok}[1]{\textcolor[rgb]{0.25,0.44,0.63}{{#1}}}
    \newcommand{\SpecialStringTok}[1]{\textcolor[rgb]{0.73,0.40,0.53}{{#1}}}
    \newcommand{\ImportTok}[1]{{#1}}
    \newcommand{\DocumentationTok}[1]{\textcolor[rgb]{0.73,0.13,0.13}{\textit{{#1}}}}
    \newcommand{\AnnotationTok}[1]{\textcolor[rgb]{0.38,0.63,0.69}{\textbf{\textit{{#1}}}}}
    \newcommand{\CommentVarTok}[1]{\textcolor[rgb]{0.38,0.63,0.69}{\textbf{\textit{{#1}}}}}
    \newcommand{\VariableTok}[1]{\textcolor[rgb]{0.10,0.09,0.49}{{#1}}}
    \newcommand{\ControlFlowTok}[1]{\textcolor[rgb]{0.00,0.44,0.13}{\textbf{{#1}}}}
    \newcommand{\OperatorTok}[1]{\textcolor[rgb]{0.40,0.40,0.40}{{#1}}}
    \newcommand{\BuiltInTok}[1]{{#1}}
    \newcommand{\ExtensionTok}[1]{{#1}}
    \newcommand{\PreprocessorTok}[1]{\textcolor[rgb]{0.74,0.48,0.00}{{#1}}}
    \newcommand{\AttributeTok}[1]{\textcolor[rgb]{0.49,0.56,0.16}{{#1}}}
    \newcommand{\InformationTok}[1]{\textcolor[rgb]{0.38,0.63,0.69}{\textbf{\textit{{#1}}}}}
    \newcommand{\WarningTok}[1]{\textcolor[rgb]{0.38,0.63,0.69}{\textbf{\textit{{#1}}}}}
    
    
    % Define a nice break command that doesn't care if a line doesn't already
    % exist.
    \def\br{\hspace*{\fill} \\* }
    % Math Jax compatability definitions
    \def\gt{>}
    \def\lt{<}
    % Document parameters
    \title{Sol\_HW\_5}
    
    
    

    % Pygments definitions
    
\makeatletter
\def\PY@reset{\let\PY@it=\relax \let\PY@bf=\relax%
    \let\PY@ul=\relax \let\PY@tc=\relax%
    \let\PY@bc=\relax \let\PY@ff=\relax}
\def\PY@tok#1{\csname PY@tok@#1\endcsname}
\def\PY@toks#1+{\ifx\relax#1\empty\else%
    \PY@tok{#1}\expandafter\PY@toks\fi}
\def\PY@do#1{\PY@bc{\PY@tc{\PY@ul{%
    \PY@it{\PY@bf{\PY@ff{#1}}}}}}}
\def\PY#1#2{\PY@reset\PY@toks#1+\relax+\PY@do{#2}}

\expandafter\def\csname PY@tok@gd\endcsname{\def\PY@tc##1{\textcolor[rgb]{0.63,0.00,0.00}{##1}}}
\expandafter\def\csname PY@tok@gu\endcsname{\let\PY@bf=\textbf\def\PY@tc##1{\textcolor[rgb]{0.50,0.00,0.50}{##1}}}
\expandafter\def\csname PY@tok@gt\endcsname{\def\PY@tc##1{\textcolor[rgb]{0.00,0.27,0.87}{##1}}}
\expandafter\def\csname PY@tok@gs\endcsname{\let\PY@bf=\textbf}
\expandafter\def\csname PY@tok@gr\endcsname{\def\PY@tc##1{\textcolor[rgb]{1.00,0.00,0.00}{##1}}}
\expandafter\def\csname PY@tok@cm\endcsname{\let\PY@it=\textit\def\PY@tc##1{\textcolor[rgb]{0.25,0.50,0.50}{##1}}}
\expandafter\def\csname PY@tok@vg\endcsname{\def\PY@tc##1{\textcolor[rgb]{0.10,0.09,0.49}{##1}}}
\expandafter\def\csname PY@tok@vi\endcsname{\def\PY@tc##1{\textcolor[rgb]{0.10,0.09,0.49}{##1}}}
\expandafter\def\csname PY@tok@vm\endcsname{\def\PY@tc##1{\textcolor[rgb]{0.10,0.09,0.49}{##1}}}
\expandafter\def\csname PY@tok@mh\endcsname{\def\PY@tc##1{\textcolor[rgb]{0.40,0.40,0.40}{##1}}}
\expandafter\def\csname PY@tok@cs\endcsname{\let\PY@it=\textit\def\PY@tc##1{\textcolor[rgb]{0.25,0.50,0.50}{##1}}}
\expandafter\def\csname PY@tok@ge\endcsname{\let\PY@it=\textit}
\expandafter\def\csname PY@tok@vc\endcsname{\def\PY@tc##1{\textcolor[rgb]{0.10,0.09,0.49}{##1}}}
\expandafter\def\csname PY@tok@il\endcsname{\def\PY@tc##1{\textcolor[rgb]{0.40,0.40,0.40}{##1}}}
\expandafter\def\csname PY@tok@go\endcsname{\def\PY@tc##1{\textcolor[rgb]{0.53,0.53,0.53}{##1}}}
\expandafter\def\csname PY@tok@cp\endcsname{\def\PY@tc##1{\textcolor[rgb]{0.74,0.48,0.00}{##1}}}
\expandafter\def\csname PY@tok@gi\endcsname{\def\PY@tc##1{\textcolor[rgb]{0.00,0.63,0.00}{##1}}}
\expandafter\def\csname PY@tok@gh\endcsname{\let\PY@bf=\textbf\def\PY@tc##1{\textcolor[rgb]{0.00,0.00,0.50}{##1}}}
\expandafter\def\csname PY@tok@ni\endcsname{\let\PY@bf=\textbf\def\PY@tc##1{\textcolor[rgb]{0.60,0.60,0.60}{##1}}}
\expandafter\def\csname PY@tok@nl\endcsname{\def\PY@tc##1{\textcolor[rgb]{0.63,0.63,0.00}{##1}}}
\expandafter\def\csname PY@tok@nn\endcsname{\let\PY@bf=\textbf\def\PY@tc##1{\textcolor[rgb]{0.00,0.00,1.00}{##1}}}
\expandafter\def\csname PY@tok@no\endcsname{\def\PY@tc##1{\textcolor[rgb]{0.53,0.00,0.00}{##1}}}
\expandafter\def\csname PY@tok@na\endcsname{\def\PY@tc##1{\textcolor[rgb]{0.49,0.56,0.16}{##1}}}
\expandafter\def\csname PY@tok@nb\endcsname{\def\PY@tc##1{\textcolor[rgb]{0.00,0.50,0.00}{##1}}}
\expandafter\def\csname PY@tok@nc\endcsname{\let\PY@bf=\textbf\def\PY@tc##1{\textcolor[rgb]{0.00,0.00,1.00}{##1}}}
\expandafter\def\csname PY@tok@nd\endcsname{\def\PY@tc##1{\textcolor[rgb]{0.67,0.13,1.00}{##1}}}
\expandafter\def\csname PY@tok@ne\endcsname{\let\PY@bf=\textbf\def\PY@tc##1{\textcolor[rgb]{0.82,0.25,0.23}{##1}}}
\expandafter\def\csname PY@tok@nf\endcsname{\def\PY@tc##1{\textcolor[rgb]{0.00,0.00,1.00}{##1}}}
\expandafter\def\csname PY@tok@si\endcsname{\let\PY@bf=\textbf\def\PY@tc##1{\textcolor[rgb]{0.73,0.40,0.53}{##1}}}
\expandafter\def\csname PY@tok@s2\endcsname{\def\PY@tc##1{\textcolor[rgb]{0.73,0.13,0.13}{##1}}}
\expandafter\def\csname PY@tok@nt\endcsname{\let\PY@bf=\textbf\def\PY@tc##1{\textcolor[rgb]{0.00,0.50,0.00}{##1}}}
\expandafter\def\csname PY@tok@nv\endcsname{\def\PY@tc##1{\textcolor[rgb]{0.10,0.09,0.49}{##1}}}
\expandafter\def\csname PY@tok@s1\endcsname{\def\PY@tc##1{\textcolor[rgb]{0.73,0.13,0.13}{##1}}}
\expandafter\def\csname PY@tok@dl\endcsname{\def\PY@tc##1{\textcolor[rgb]{0.73,0.13,0.13}{##1}}}
\expandafter\def\csname PY@tok@ch\endcsname{\let\PY@it=\textit\def\PY@tc##1{\textcolor[rgb]{0.25,0.50,0.50}{##1}}}
\expandafter\def\csname PY@tok@m\endcsname{\def\PY@tc##1{\textcolor[rgb]{0.40,0.40,0.40}{##1}}}
\expandafter\def\csname PY@tok@gp\endcsname{\let\PY@bf=\textbf\def\PY@tc##1{\textcolor[rgb]{0.00,0.00,0.50}{##1}}}
\expandafter\def\csname PY@tok@sh\endcsname{\def\PY@tc##1{\textcolor[rgb]{0.73,0.13,0.13}{##1}}}
\expandafter\def\csname PY@tok@ow\endcsname{\let\PY@bf=\textbf\def\PY@tc##1{\textcolor[rgb]{0.67,0.13,1.00}{##1}}}
\expandafter\def\csname PY@tok@sx\endcsname{\def\PY@tc##1{\textcolor[rgb]{0.00,0.50,0.00}{##1}}}
\expandafter\def\csname PY@tok@bp\endcsname{\def\PY@tc##1{\textcolor[rgb]{0.00,0.50,0.00}{##1}}}
\expandafter\def\csname PY@tok@c1\endcsname{\let\PY@it=\textit\def\PY@tc##1{\textcolor[rgb]{0.25,0.50,0.50}{##1}}}
\expandafter\def\csname PY@tok@fm\endcsname{\def\PY@tc##1{\textcolor[rgb]{0.00,0.00,1.00}{##1}}}
\expandafter\def\csname PY@tok@o\endcsname{\def\PY@tc##1{\textcolor[rgb]{0.40,0.40,0.40}{##1}}}
\expandafter\def\csname PY@tok@kc\endcsname{\let\PY@bf=\textbf\def\PY@tc##1{\textcolor[rgb]{0.00,0.50,0.00}{##1}}}
\expandafter\def\csname PY@tok@c\endcsname{\let\PY@it=\textit\def\PY@tc##1{\textcolor[rgb]{0.25,0.50,0.50}{##1}}}
\expandafter\def\csname PY@tok@mf\endcsname{\def\PY@tc##1{\textcolor[rgb]{0.40,0.40,0.40}{##1}}}
\expandafter\def\csname PY@tok@err\endcsname{\def\PY@bc##1{\setlength{\fboxsep}{0pt}\fcolorbox[rgb]{1.00,0.00,0.00}{1,1,1}{\strut ##1}}}
\expandafter\def\csname PY@tok@mb\endcsname{\def\PY@tc##1{\textcolor[rgb]{0.40,0.40,0.40}{##1}}}
\expandafter\def\csname PY@tok@ss\endcsname{\def\PY@tc##1{\textcolor[rgb]{0.10,0.09,0.49}{##1}}}
\expandafter\def\csname PY@tok@sr\endcsname{\def\PY@tc##1{\textcolor[rgb]{0.73,0.40,0.53}{##1}}}
\expandafter\def\csname PY@tok@mo\endcsname{\def\PY@tc##1{\textcolor[rgb]{0.40,0.40,0.40}{##1}}}
\expandafter\def\csname PY@tok@kd\endcsname{\let\PY@bf=\textbf\def\PY@tc##1{\textcolor[rgb]{0.00,0.50,0.00}{##1}}}
\expandafter\def\csname PY@tok@mi\endcsname{\def\PY@tc##1{\textcolor[rgb]{0.40,0.40,0.40}{##1}}}
\expandafter\def\csname PY@tok@kn\endcsname{\let\PY@bf=\textbf\def\PY@tc##1{\textcolor[rgb]{0.00,0.50,0.00}{##1}}}
\expandafter\def\csname PY@tok@cpf\endcsname{\let\PY@it=\textit\def\PY@tc##1{\textcolor[rgb]{0.25,0.50,0.50}{##1}}}
\expandafter\def\csname PY@tok@kr\endcsname{\let\PY@bf=\textbf\def\PY@tc##1{\textcolor[rgb]{0.00,0.50,0.00}{##1}}}
\expandafter\def\csname PY@tok@s\endcsname{\def\PY@tc##1{\textcolor[rgb]{0.73,0.13,0.13}{##1}}}
\expandafter\def\csname PY@tok@kp\endcsname{\def\PY@tc##1{\textcolor[rgb]{0.00,0.50,0.00}{##1}}}
\expandafter\def\csname PY@tok@w\endcsname{\def\PY@tc##1{\textcolor[rgb]{0.73,0.73,0.73}{##1}}}
\expandafter\def\csname PY@tok@kt\endcsname{\def\PY@tc##1{\textcolor[rgb]{0.69,0.00,0.25}{##1}}}
\expandafter\def\csname PY@tok@sc\endcsname{\def\PY@tc##1{\textcolor[rgb]{0.73,0.13,0.13}{##1}}}
\expandafter\def\csname PY@tok@sb\endcsname{\def\PY@tc##1{\textcolor[rgb]{0.73,0.13,0.13}{##1}}}
\expandafter\def\csname PY@tok@sa\endcsname{\def\PY@tc##1{\textcolor[rgb]{0.73,0.13,0.13}{##1}}}
\expandafter\def\csname PY@tok@k\endcsname{\let\PY@bf=\textbf\def\PY@tc##1{\textcolor[rgb]{0.00,0.50,0.00}{##1}}}
\expandafter\def\csname PY@tok@se\endcsname{\let\PY@bf=\textbf\def\PY@tc##1{\textcolor[rgb]{0.73,0.40,0.13}{##1}}}
\expandafter\def\csname PY@tok@sd\endcsname{\let\PY@it=\textit\def\PY@tc##1{\textcolor[rgb]{0.73,0.13,0.13}{##1}}}

\def\PYZbs{\char`\\}
\def\PYZus{\char`\_}
\def\PYZob{\char`\{}
\def\PYZcb{\char`\}}
\def\PYZca{\char`\^}
\def\PYZam{\char`\&}
\def\PYZlt{\char`\<}
\def\PYZgt{\char`\>}
\def\PYZsh{\char`\#}
\def\PYZpc{\char`\%}
\def\PYZdl{\char`\$}
\def\PYZhy{\char`\-}
\def\PYZsq{\char`\'}
\def\PYZdq{\char`\"}
\def\PYZti{\char`\~}
% for compatibility with earlier versions
\def\PYZat{@}
\def\PYZlb{[}
\def\PYZrb{]}
\makeatother


    % Exact colors from NB
    \definecolor{incolor}{rgb}{0.0, 0.0, 0.5}
    \definecolor{outcolor}{rgb}{0.545, 0.0, 0.0}



    
    % Prevent overflowing lines due to hard-to-break entities
    \sloppy 
    % Setup hyperref package
    \hypersetup{
      breaklinks=true,  % so long urls are correctly broken across lines
      colorlinks=true,
      urlcolor=urlcolor,
      linkcolor=linkcolor,
      citecolor=citecolor,
      }
    % Slightly bigger margins than the latex defaults
    
    \geometry{verbose,tmargin=1in,bmargin=1in,lmargin=1in,rmargin=1in}
    
    

    \begin{document}
    
    
    \maketitle
    
    

    
    \hypertarget{mecuxe1nica-cuuxe1ntica-1--201810}{%
\section{Mecánica Cuántica 1
-201810}\label{mecuxe1nica-cuuxe1ntica-1--201810}}

\hypertarget{tarea-2---soluciuxf3n}{%
\subsection{Tarea \# 2 - Solución}\label{tarea-2---soluciuxf3n}}

Elaborada por Daniel Forero.

    \begin{Verbatim}[commandchars=\\\{\}]
{\color{incolor}In [{\color{incolor}1}]:} \PY{k+kn}{from} \PY{n+nn}{sympy} \PY{k+kn}{import} \PY{o}{*}
        \PY{n}{init\PYZus{}printing}\PY{p}{(}\PY{p}{)}
        \PY{k+kn}{from} \PY{n+nn}{IPython.display} \PY{k+kn}{import} \PY{n}{display}
\end{Verbatim}


    \hypertarget{section}{%
\subsubsection{2.}\label{section}}

Tenemos un espacio vectorial bidimensional expandido por
\(\{\lvert 1\rangle, \lvert 2\rangle\}\).

Además, se nos da la matriz \(\sigma_2\) (segunda matriz de Pauli), para
comprobar su hermicidad basta con comprobar
\[\sigma_2=\sigma^\dagger_2.\]

    \begin{Verbatim}[commandchars=\\\{\}]
{\color{incolor}In [{\color{incolor}2}]:} \PY{k+kn}{from} \PY{n+nn}{sympy.physics.matrices} \PY{k+kn}{import} \PY{n}{msigma}
        \PY{n}{sigma\PYZus{}2} \PY{o}{=} \PY{n}{msigma}\PY{p}{(}\PY{l+m+mi}{2}\PY{p}{)}
        
        \PY{n}{display}\PY{p}{(}\PY{n}{sigma\PYZus{}2}\PY{o}{.}\PY{n}{transpose}\PY{p}{(}\PY{p}{)}\PY{o}{.}\PY{n}{conjugate}\PY{p}{(}\PY{p}{)}\PY{p}{)}
        \PY{n}{display}\PY{p}{(}\PY{n}{sigma\PYZus{}2}\PY{o}{.}\PY{n}{transpose}\PY{p}{(}\PY{p}{)}\PY{o}{.}\PY{n}{conjugate}\PY{p}{(}\PY{p}{)}\PY{o}{==}\PY{n}{sigma\PYZus{}2}\PY{p}{)}
\end{Verbatim}


    $$\left[\begin{matrix}0 & - i\\i & 0\end{matrix}\right]$$

    
    
    \begin{verbatim}
True
    \end{verbatim}

    
    Como vemos, la relación se cumple. Además podemos notar que
\(\sigma_2^\dagger\sigma_2 = 1\), es decir \(\sigma_2^2 = 1\) esta
relación se cumple para todas las matrices \(\sigma_i,\ i=1,2,3\).
Concluimos que la matriz es hermítica (y unitaria), el observable
asociado a estas matrices es el spin \(S_i = \frac{\hbar}{2}\sigma_i\).

A continuación calculamos los vectores y valores propios.

    \begin{Verbatim}[commandchars=\\\{\}]
{\color{incolor}In [{\color{incolor}3}]:} \PY{n}{evect} \PY{o}{=} \PY{n}{sigma\PYZus{}2}\PY{o}{.}\PY{n}{eigenvects}\PY{p}{(}\PY{p}{)}
        \PY{n}{display}\PY{p}{(}\PY{n}{evect}\PY{p}{)}
        \PY{n}{ket\PYZus{}p} \PY{o}{=} \PY{n}{evect}\PY{p}{[}\PY{l+m+mi}{0}\PY{p}{]}\PY{p}{[}\PY{o}{\PYZhy{}}\PY{l+m+mi}{1}\PY{p}{]}\PY{p}{[}\PY{l+m+mi}{0}\PY{p}{]}
        \PY{n}{ket\PYZus{}p}\PY{o}{/}\PY{o}{=}\PY{n}{ket\PYZus{}p}\PY{o}{.}\PY{n}{norm}\PY{p}{(}\PY{p}{)}
        \PY{n}{ket\PYZus{}m} \PY{o}{=} \PY{n}{evect}\PY{p}{[}\PY{l+m+mi}{1}\PY{p}{]}\PY{p}{[}\PY{o}{\PYZhy{}}\PY{l+m+mi}{1}\PY{p}{]}\PY{p}{[}\PY{l+m+mi}{0}\PY{p}{]}
        \PY{n}{ket\PYZus{}m}\PY{o}{/}\PY{o}{=}\PY{n}{ket\PYZus{}m}\PY{o}{.}\PY{n}{norm}\PY{p}{(}\PY{p}{)}
        \PY{n}{ket\PYZus{}1} \PY{o}{=} \PY{n}{Matrix}\PY{p}{(}\PY{p}{[}\PY{p}{[}\PY{l+m+mi}{1}\PY{p}{]}\PY{p}{,}\PY{p}{[}\PY{l+m+mi}{0}\PY{p}{]}\PY{p}{]}\PY{p}{)}
        \PY{n}{ket\PYZus{}1}\PY{o}{/}\PY{o}{=}\PY{n}{ket\PYZus{}1}\PY{o}{.}\PY{n}{norm}\PY{p}{(}\PY{p}{)}
        \PY{n}{ket\PYZus{}2} \PY{o}{=} \PY{n}{Matrix}\PY{p}{(}\PY{p}{[}\PY{p}{[}\PY{l+m+mi}{0}\PY{p}{]}\PY{p}{,}\PY{p}{[}\PY{l+m+mi}{1}\PY{p}{]}\PY{p}{]}\PY{p}{)}
        \PY{n}{ket\PYZus{}2}\PY{o}{/}\PY{o}{=}\PY{n}{ket\PYZus{}2}\PY{o}{.}\PY{n}{norm}\PY{p}{(}\PY{p}{)}
        \PY{n}{base\PYZus{}1} \PY{o}{=} \PY{p}{[}\PY{n}{ket\PYZus{}1}\PY{p}{,} \PY{n}{ket\PYZus{}2}\PY{p}{]}
        \PY{n}{evects\PYZus{}1} \PY{o}{=} \PY{p}{[}\PY{n}{ket\PYZus{}p}\PY{p}{,} \PY{n}{ket\PYZus{}m}\PY{p}{]}
\end{Verbatim}


    $$\left [ \left ( -1, \quad 1, \quad \left [ \left[\begin{matrix}i\\1\end{matrix}\right]\right ]\right ), \quad \left ( 1, \quad 1, \quad \left [ \left[\begin{matrix}- i\\1\end{matrix}\right]\right ]\right )\right ]$$

    
    La diagonalización es sencilla, se espera hecha a mano. Vemos que la
matriz tiene valores propios son \(\lambda_\pm = \pm 1\), con
multiplicidad 1 (segunda entrada) y con los vectores propios mostrados
\(\{\lvert -\rangle,\ \lvert +\rangle\},\) respectivamente. En la base
dada, podemos expresar:
\[\lvert +\rangle = \frac{1}{\sqrt{2}}(i\lvert1\rangle + \lvert2\rangle)\]
\[\lvert -\rangle = \frac{1}{\sqrt{2}}(-i\lvert1\rangle + \lvert2\rangle)\]

Los proyectores a estos vectores propios son
\(P_\pm = \lvert\pm\rangle\langle\pm\lvert\). Para calcular las matrices
asociadas (en la base \(\{\lvert 1\rangle, \lvert 2\rangle\}\))
solamente aplicamos
\(P_{\pm,ij} = \langle i \rvert P_\pm \lvert j \rangle\).

    \begin{Verbatim}[commandchars=\\\{\}]
{\color{incolor}In [{\color{incolor}4}]:} \PY{k}{def} \PY{n+nf}{bra\PYZus{}ket}\PY{p}{(}\PY{n}{vec1}\PY{p}{,} \PY{n}{vec2}\PY{p}{)}\PY{p}{:}
            \PY{k}{return} \PY{p}{(}\PY{p}{(}\PY{n}{vec1}\PY{o}{.}\PY{n}{transpose}\PY{p}{(}\PY{p}{)}\PY{o}{.}\PY{n}{conjugate}\PY{p}{(}\PY{p}{)}\PY{p}{)}\PY{o}{*}\PY{n}{vec2}\PY{p}{)}\PY{p}{[}\PY{l+m+mi}{0}\PY{p}{]}
        \PY{k}{def} \PY{n+nf}{build\PYZus{}projectors}\PY{p}{(}\PY{n}{base}\PY{p}{,}\PY{n}{pr}\PY{p}{)}\PY{p}{:}
            
            \PY{n}{P} \PY{o}{=} \PY{n}{zeros}\PY{p}{(}\PY{n+nb}{len}\PY{p}{(}\PY{n}{base}\PY{p}{)}\PY{p}{)}
            \PY{k}{for} \PY{n}{i} \PY{o+ow}{in} \PY{n+nb}{range}\PY{p}{(}\PY{n+nb}{len}\PY{p}{(}\PY{n}{base}\PY{p}{)}\PY{p}{)}\PY{p}{:}
                \PY{k}{for} \PY{n}{k} \PY{o+ow}{in} \PY{n+nb}{range}\PY{p}{(}\PY{n+nb}{len}\PY{p}{(}\PY{n}{base}\PY{p}{)}\PY{p}{)}\PY{p}{:}
                    \PY{n}{P}\PY{p}{[}\PY{n}{i}\PY{p}{,}\PY{n}{k}\PY{p}{]} \PY{o}{=} \PY{n}{bra\PYZus{}ket}\PY{p}{(}\PY{n}{base}\PY{p}{[}\PY{n}{i}\PY{p}{]}\PY{p}{,} \PY{n}{pr}\PY{p}{)}\PY{o}{*}\PY{n}{bra\PYZus{}ket}\PY{p}{(}\PY{n}{pr}\PY{p}{,} \PY{n}{base}\PY{p}{[}\PY{n}{k}\PY{p}{]}\PY{p}{)}
            \PY{k}{return} \PY{n}{P}
\end{Verbatim}


    \begin{Verbatim}[commandchars=\\\{\}]
{\color{incolor}In [{\color{incolor}5}]:} \PY{n}{Ps} \PY{o}{=} \PY{p}{[}\PY{p}{]}
        \PY{k}{for} \PY{n}{ket} \PY{o+ow}{in} \PY{n}{evects\PYZus{}1}\PY{p}{:}
            \PY{n}{P} \PY{o}{=}\PY{n}{build\PYZus{}projectors}\PY{p}{(}\PY{n}{base\PYZus{}1}\PY{p}{,} \PY{n}{ket}\PY{p}{)}
            \PY{n}{Ps}\PY{o}{.}\PY{n}{append}\PY{p}{(}\PY{n}{P}\PY{p}{)}
            \PY{n}{display}\PY{p}{(}\PY{l+s+s1}{\PYZsq{}}\PY{l+s+s1}{For eigket}\PY{l+s+s1}{\PYZsq{}}\PY{p}{,}\PY{n}{ket}\PY{p}{)}
            \PY{n}{display} \PY{p}{(}\PY{l+s+s1}{\PYZsq{}}\PY{l+s+s1}{Matrix}\PY{l+s+s1}{\PYZsq{}}\PY{p}{,}\PY{n}{P}\PY{p}{,}\PY{l+s+s1}{\PYZsq{}}\PY{l+s+s1}{Projector?}\PY{l+s+s1}{\PYZsq{}}\PY{p}{,}\PY{n}{P}\PY{o}{*}\PY{o}{*}\PY{l+m+mi}{2}\PY{o}{==}\PY{n}{P}\PY{p}{,}\PY{n}{P}\PY{o}{*}\PY{o}{*}\PY{l+m+mi}{2}\PY{p}{,}\PY{n}{P} \PY{p}{)}
            
        \PY{n}{display}\PY{p}{(}\PY{l+s+s1}{\PYZsq{}}\PY{l+s+s1}{Closed?}\PY{l+s+s1}{\PYZsq{}}\PY{p}{,}\PY{n+nb}{reduce}\PY{p}{(}\PY{k}{lambda} \PY{n}{x}\PY{p}{,} \PY{n}{y}\PY{p}{:} \PY{n}{x}\PY{o}{+}\PY{n}{y}\PY{p}{,} \PY{n}{Ps}\PY{p}{)}\PY{o}{==}\PY{n}{eye}\PY{p}{(}\PY{l+m+mi}{2}\PY{p}{)}\PY{p}{)}
        \PY{n}{display}\PY{p}{(}\PY{l+s+s1}{\PYZsq{}}\PY{l+s+s1}{Orthogonal?}\PY{l+s+s1}{\PYZsq{}}\PY{p}{,}\PY{n+nb}{reduce}\PY{p}{(}\PY{k}{lambda} \PY{n}{x}\PY{p}{,} \PY{n}{y}\PY{p}{:} \PY{n}{x}\PY{o}{*}\PY{n}{y}\PY{p}{,} \PY{n}{Ps}\PY{p}{)}\PY{o}{==}\PY{n}{zeros}\PY{p}{(}\PY{l+m+mi}{2}\PY{p}{)}\PY{p}{)}
\end{Verbatim}


    
    \begin{verbatim}
'For eigket'
    \end{verbatim}

    
    $$\left[\begin{matrix}\frac{\sqrt{2} i}{2}\\\frac{\sqrt{2}}{2}\end{matrix}\right]$$

    
    
    \begin{verbatim}
'Matrix'
    \end{verbatim}

    
    $$\left[\begin{matrix}\frac{1}{2} & \frac{i}{2}\\- \frac{i}{2} & \frac{1}{2}\end{matrix}\right]$$

    
    
    \begin{verbatim}
'Projector?'
    \end{verbatim}

    
    
    \begin{verbatim}
True
    \end{verbatim}

    
    $$\left[\begin{matrix}\frac{1}{2} & \frac{i}{2}\\- \frac{i}{2} & \frac{1}{2}\end{matrix}\right]$$

    
    $$\left[\begin{matrix}\frac{1}{2} & \frac{i}{2}\\- \frac{i}{2} & \frac{1}{2}\end{matrix}\right]$$

    
    
    \begin{verbatim}
'For eigket'
    \end{verbatim}

    
    $$\left[\begin{matrix}- \frac{\sqrt{2} i}{2}\\\frac{\sqrt{2}}{2}\end{matrix}\right]$$

    
    
    \begin{verbatim}
'Matrix'
    \end{verbatim}

    
    $$\left[\begin{matrix}\frac{1}{2} & - \frac{i}{2}\\\frac{i}{2} & \frac{1}{2}\end{matrix}\right]$$

    
    
    \begin{verbatim}
'Projector?'
    \end{verbatim}

    
    
    \begin{verbatim}
True
    \end{verbatim}

    
    $$\left[\begin{matrix}\frac{1}{2} & - \frac{i}{2}\\\frac{i}{2} & \frac{1}{2}\end{matrix}\right]$$

    
    $$\left[\begin{matrix}\frac{1}{2} & - \frac{i}{2}\\\frac{i}{2} & \frac{1}{2}\end{matrix}\right]$$

    
    
    \begin{verbatim}
'Closed?'
    \end{verbatim}

    
    
    \begin{verbatim}
True
    \end{verbatim}

    
    
    \begin{verbatim}
'Orthogonal?'
    \end{verbatim}

    
    
    \begin{verbatim}
True
    \end{verbatim}

    
    \begin{Verbatim}[commandchars=\\\{\}]
{\color{incolor}In [{\color{incolor}6}]:} \PY{k}{def} \PY{n+nf}{do\PYZus{}hw}\PY{p}{(}\PY{n}{matrix}\PY{p}{,} \PY{n}{base}\PY{p}{)}\PY{p}{:}
            \PY{n}{evect} \PY{o}{=} \PY{n}{matrix}\PY{o}{.}\PY{n}{eigenvects}\PY{p}{(}\PY{p}{)}
            \PY{n}{display}\PY{p}{(}\PY{l+s+s1}{\PYZsq{}}\PY{l+s+s1}{Eigenvalues, multiplicity, eigenvectors (no norm.)}\PY{l+s+s1}{\PYZsq{}}\PY{p}{,}\PY{n}{evect}\PY{p}{)}
            \PY{n}{evects} \PY{o}{=} \PY{p}{[}\PY{p}{]}
            \PY{k}{for} \PY{n}{i} \PY{o+ow}{in} \PY{n+nb}{range}\PY{p}{(}\PY{n+nb}{len}\PY{p}{(}\PY{n}{base}\PY{p}{)}\PY{p}{)}\PY{p}{:}
                \PY{n}{k} \PY{o}{=}\PY{n}{evect}\PY{p}{[}\PY{n}{i}\PY{p}{]}\PY{p}{[}\PY{o}{\PYZhy{}}\PY{l+m+mi}{1}\PY{p}{]}\PY{p}{[}\PY{l+m+mi}{0}\PY{p}{]}
                \PY{n}{k}\PY{o}{/}\PY{o}{=}\PY{n}{k}\PY{o}{.}\PY{n}{norm}\PY{p}{(}\PY{p}{)}
                \PY{n}{evects}\PY{o}{.}\PY{n}{append}\PY{p}{(}\PY{n}{k}\PY{p}{)}
            \PY{n}{Ps} \PY{o}{=} \PY{p}{[}\PY{p}{]}
            \PY{k}{for} \PY{n}{ket} \PY{o+ow}{in} \PY{n}{evects}\PY{p}{:}
                \PY{n}{P} \PY{o}{=}\PY{n}{build\PYZus{}projectors}\PY{p}{(}\PY{n}{base}\PY{p}{,} \PY{n}{ket}\PY{p}{)}
                \PY{n}{Ps}\PY{o}{.}\PY{n}{append}\PY{p}{(}\PY{n}{P}\PY{p}{)}
                \PY{n}{display}\PY{p}{(}\PY{l+s+s1}{\PYZsq{}}\PY{l+s+s1}{For eigket}\PY{l+s+s1}{\PYZsq{}}\PY{p}{,}\PY{n}{ket}\PY{p}{)}
                \PY{n}{display} \PY{p}{(}\PY{l+s+s1}{\PYZsq{}}\PY{l+s+s1}{Matrix}\PY{l+s+s1}{\PYZsq{}}\PY{p}{,}\PY{n}{P}\PY{p}{,}\PY{l+s+s1}{\PYZsq{}}\PY{l+s+s1}{Projector?}\PY{l+s+s1}{\PYZsq{}}\PY{p}{,}\PY{n}{P}\PY{o}{*}\PY{o}{*}\PY{l+m+mi}{2}\PY{o}{==}\PY{n}{P}\PY{p}{,}\PY{n}{P}\PY{o}{*}\PY{o}{*}\PY{l+m+mi}{2}\PY{p}{,}\PY{n}{P} \PY{p}{)}
            
            \PY{n}{display}\PY{p}{(}\PY{l+s+s1}{\PYZsq{}}\PY{l+s+s1}{Closed?}\PY{l+s+s1}{\PYZsq{}}\PY{p}{,}\PY{n+nb}{reduce}\PY{p}{(}\PY{k}{lambda} \PY{n}{x}\PY{p}{,} \PY{n}{y}\PY{p}{:} \PY{n}{x}\PY{o}{+}\PY{n}{y}\PY{p}{,} \PY{n}{Ps}\PY{p}{)}\PY{o}{==}\PY{n}{eye}\PY{p}{(}\PY{n+nb}{len}\PY{p}{(}\PY{n}{base}\PY{p}{)}\PY{p}{)}\PY{p}{)}
            \PY{n}{display}\PY{p}{(}\PY{l+s+s1}{\PYZsq{}}\PY{l+s+s1}{Orthogonal?}\PY{l+s+s1}{\PYZsq{}}\PY{p}{,}\PY{n+nb}{reduce}\PY{p}{(}\PY{k}{lambda} \PY{n}{x}\PY{p}{,} \PY{n}{y}\PY{p}{:} \PY{n}{x}\PY{o}{*}\PY{n}{y}\PY{p}{,} \PY{n}{Ps}\PY{p}{)}\PY{o}{==}\PY{n}{zeros}\PY{p}{(}\PY{n+nb}{len}\PY{p}{(}\PY{n}{base}\PY{p}{)}\PY{p}{)}\PY{p}{)}
\end{Verbatim}


    \begin{Verbatim}[commandchars=\\\{\}]
{\color{incolor}In [{\color{incolor}7}]:} \PY{n}{second\PYZus{}mat} \PY{o}{=} \PY{n}{Matrix}\PY{p}{(}\PY{p}{[}\PY{p}{[}\PY{l+m+mi}{2}\PY{p}{,} \PY{n}{sqrt}\PY{p}{(}\PY{l+m+mi}{2}\PY{p}{)}\PY{o}{*}\PY{n}{I}\PY{p}{]}\PY{p}{,}\PY{p}{[}\PY{o}{\PYZhy{}}\PY{n}{sqrt}\PY{p}{(}\PY{l+m+mi}{2}\PY{p}{)}\PY{o}{*}\PY{n}{I}\PY{p}{,}\PY{l+m+mi}{3}\PY{p}{]}\PY{p}{]}\PY{p}{)}
        \PY{n}{do\PYZus{}hw}\PY{p}{(}\PY{n}{second\PYZus{}mat}\PY{p}{,} \PY{n}{base\PYZus{}1}\PY{p}{)}
\end{Verbatim}


    
    \begin{verbatim}
'Eigenvalues, multiplicity, eigenvectors (no norm.)'
    \end{verbatim}

    
    $$\left [ \left ( 1, \quad 1, \quad \left [ \left[\begin{matrix}- \sqrt{2} i\\1\end{matrix}\right]\right ]\right ), \quad \left ( 4, \quad 1, \quad \left [ \left[\begin{matrix}\frac{\sqrt{2} i}{2}\\1\end{matrix}\right]\right ]\right )\right ]$$

    
    
    \begin{verbatim}
'For eigket'
    \end{verbatim}

    
    $$\left[\begin{matrix}- \frac{\sqrt{6} i}{3}\\\frac{\sqrt{3}}{3}\end{matrix}\right]$$

    
    
    \begin{verbatim}
'Matrix'
    \end{verbatim}

    
    $$\left[\begin{matrix}\frac{2}{3} & - \frac{\sqrt{2} i}{3}\\\frac{\sqrt{2} i}{3} & \frac{1}{3}\end{matrix}\right]$$

    
    
    \begin{verbatim}
'Projector?'
    \end{verbatim}

    
    
    \begin{verbatim}
True
    \end{verbatim}

    
    $$\left[\begin{matrix}\frac{2}{3} & - \frac{\sqrt{2} i}{3}\\\frac{\sqrt{2} i}{3} & \frac{1}{3}\end{matrix}\right]$$

    
    $$\left[\begin{matrix}\frac{2}{3} & - \frac{\sqrt{2} i}{3}\\\frac{\sqrt{2} i}{3} & \frac{1}{3}\end{matrix}\right]$$

    
    
    \begin{verbatim}
'For eigket'
    \end{verbatim}

    
    $$\left[\begin{matrix}\frac{\sqrt{3} i}{3}\\\frac{\sqrt{6}}{3}\end{matrix}\right]$$

    
    
    \begin{verbatim}
'Matrix'
    \end{verbatim}

    
    $$\left[\begin{matrix}\frac{1}{3} & \frac{\sqrt{2} i}{3}\\- \frac{\sqrt{2} i}{3} & \frac{2}{3}\end{matrix}\right]$$

    
    
    \begin{verbatim}
'Projector?'
    \end{verbatim}

    
    
    \begin{verbatim}
True
    \end{verbatim}

    
    $$\left[\begin{matrix}\frac{1}{3} & \frac{\sqrt{2} i}{3}\\- \frac{\sqrt{2} i}{3} & \frac{2}{3}\end{matrix}\right]$$

    
    $$\left[\begin{matrix}\frac{1}{3} & \frac{\sqrt{2} i}{3}\\- \frac{\sqrt{2} i}{3} & \frac{2}{3}\end{matrix}\right]$$

    
    
    \begin{verbatim}
'Closed?'
    \end{verbatim}

    
    
    \begin{verbatim}
True
    \end{verbatim}

    
    
    \begin{verbatim}
'Orthogonal?'
    \end{verbatim}

    
    
    \begin{verbatim}
True
    \end{verbatim}

    
    \begin{Verbatim}[commandchars=\\\{\}]
{\color{incolor}In [{\color{incolor}8}]:} \PY{n}{h} \PY{o}{=} \PY{n}{Symbol}\PY{p}{(}\PY{l+s+s1}{\PYZsq{}}\PY{l+s+s1}{hbar}\PY{l+s+s1}{\PYZsq{}}\PY{p}{)}
        \PY{n}{third\PYZus{}mat} \PY{o}{=} \PY{p}{(}\PY{n}{h}\PY{o}{/}\PY{p}{(}\PY{l+m+mi}{2}\PY{o}{*}\PY{n}{I}\PY{p}{)}\PY{p}{)}\PY{o}{*}\PY{n}{Matrix}\PY{p}{(}\PY{p}{[}\PY{p}{[}\PY{l+m+mi}{0}\PY{p}{,}\PY{n}{sqrt}\PY{p}{(}\PY{l+m+mi}{2}\PY{p}{)}\PY{p}{,} \PY{l+m+mi}{0}\PY{p}{]}\PY{p}{,} \PY{p}{[}\PY{o}{\PYZhy{}}\PY{n}{sqrt}\PY{p}{(}\PY{l+m+mi}{2}\PY{p}{)}\PY{p}{,}\PY{l+m+mi}{0}\PY{p}{,}\PY{n}{sqrt}\PY{p}{(}\PY{l+m+mi}{2}\PY{p}{)}\PY{p}{]}\PY{p}{,} \PY{p}{[}\PY{l+m+mi}{0}\PY{p}{,}\PY{o}{\PYZhy{}}\PY{n}{sqrt}\PY{p}{(}\PY{l+m+mi}{2}\PY{p}{)}\PY{p}{,}\PY{l+m+mi}{0}\PY{p}{]}\PY{p}{]}\PY{p}{)}
        \PY{n}{ket\PYZus{}1} \PY{o}{=} \PY{n}{Matrix}\PY{p}{(}\PY{p}{[}\PY{p}{[}\PY{l+m+mi}{1}\PY{p}{]}\PY{p}{,}\PY{p}{[}\PY{l+m+mi}{0}\PY{p}{]}\PY{p}{,}\PY{p}{[}\PY{l+m+mi}{0}\PY{p}{]}\PY{p}{]}\PY{p}{)}
        \PY{n}{ket\PYZus{}2} \PY{o}{=} \PY{n}{Matrix}\PY{p}{(}\PY{p}{[}\PY{p}{[}\PY{l+m+mi}{0}\PY{p}{]}\PY{p}{,}\PY{p}{[}\PY{l+m+mi}{1}\PY{p}{]}\PY{p}{,}\PY{p}{[}\PY{l+m+mi}{0}\PY{p}{]}\PY{p}{]}\PY{p}{)}
        \PY{n}{ket\PYZus{}3} \PY{o}{=} \PY{n}{Matrix}\PY{p}{(}\PY{p}{[}\PY{p}{[}\PY{l+m+mi}{0}\PY{p}{]}\PY{p}{,}\PY{p}{[}\PY{l+m+mi}{0}\PY{p}{]}\PY{p}{,}\PY{p}{[}\PY{l+m+mi}{1}\PY{p}{]}\PY{p}{]}\PY{p}{)}
        \PY{n}{base\PYZus{}2} \PY{o}{=} \PY{p}{[}\PY{n}{ket\PYZus{}1}\PY{p}{,} \PY{n}{ket\PYZus{}2}\PY{p}{,} \PY{n}{ket\PYZus{}3}\PY{p}{]}
        \PY{n}{do\PYZus{}hw}\PY{p}{(}\PY{n}{third\PYZus{}mat}\PY{p}{,} \PY{n}{base\PYZus{}2}\PY{p}{)}
\end{Verbatim}


    
    \begin{verbatim}
'Eigenvalues, multiplicity, eigenvectors (no norm.)'
    \end{verbatim}

    
    $$\left [ \left ( 0, \quad 1, \quad \left [ \left[\begin{matrix}1\\0\\1\end{matrix}\right]\right ]\right ), \quad \left ( - \hbar, \quad 1, \quad \left [ \left[\begin{matrix}-1\\\sqrt{2} i\\1\end{matrix}\right]\right ]\right ), \quad \left ( \hbar, \quad 1, \quad \left [ \left[\begin{matrix}-1\\- \sqrt{2} i\\1\end{matrix}\right]\right ]\right )\right ]$$

    
    
    \begin{verbatim}
'For eigket'
    \end{verbatim}

    
    $$\left[\begin{matrix}\frac{\sqrt{2}}{2}\\0\\\frac{\sqrt{2}}{2}\end{matrix}\right]$$

    
    
    \begin{verbatim}
'Matrix'
    \end{verbatim}

    
    $$\left[\begin{matrix}\frac{1}{2} & 0 & \frac{1}{2}\\0 & 0 & 0\\\frac{1}{2} & 0 & \frac{1}{2}\end{matrix}\right]$$

    
    
    \begin{verbatim}
'Projector?'
    \end{verbatim}

    
    
    \begin{verbatim}
True
    \end{verbatim}

    
    $$\left[\begin{matrix}\frac{1}{2} & 0 & \frac{1}{2}\\0 & 0 & 0\\\frac{1}{2} & 0 & \frac{1}{2}\end{matrix}\right]$$

    
    $$\left[\begin{matrix}\frac{1}{2} & 0 & \frac{1}{2}\\0 & 0 & 0\\\frac{1}{2} & 0 & \frac{1}{2}\end{matrix}\right]$$

    
    
    \begin{verbatim}
'For eigket'
    \end{verbatim}

    
    $$\left[\begin{matrix}- \frac{1}{2}\\\frac{\sqrt{2} i}{2}\\\frac{1}{2}\end{matrix}\right]$$

    
    
    \begin{verbatim}
'Matrix'
    \end{verbatim}

    
    $$\left[\begin{matrix}\frac{1}{4} & \frac{\sqrt{2} i}{4} & - \frac{1}{4}\\- \frac{\sqrt{2} i}{4} & \frac{1}{2} & \frac{\sqrt{2} i}{4}\\- \frac{1}{4} & - \frac{\sqrt{2} i}{4} & \frac{1}{4}\end{matrix}\right]$$

    
    
    \begin{verbatim}
'Projector?'
    \end{verbatim}

    
    
    \begin{verbatim}
True
    \end{verbatim}

    
    $$\left[\begin{matrix}\frac{1}{4} & \frac{\sqrt{2} i}{4} & - \frac{1}{4}\\- \frac{\sqrt{2} i}{4} & \frac{1}{2} & \frac{\sqrt{2} i}{4}\\- \frac{1}{4} & - \frac{\sqrt{2} i}{4} & \frac{1}{4}\end{matrix}\right]$$

    
    $$\left[\begin{matrix}\frac{1}{4} & \frac{\sqrt{2} i}{4} & - \frac{1}{4}\\- \frac{\sqrt{2} i}{4} & \frac{1}{2} & \frac{\sqrt{2} i}{4}\\- \frac{1}{4} & - \frac{\sqrt{2} i}{4} & \frac{1}{4}\end{matrix}\right]$$

    
    
    \begin{verbatim}
'For eigket'
    \end{verbatim}

    
    $$\left[\begin{matrix}- \frac{1}{2}\\- \frac{\sqrt{2} i}{2}\\\frac{1}{2}\end{matrix}\right]$$

    
    
    \begin{verbatim}
'Matrix'
    \end{verbatim}

    
    $$\left[\begin{matrix}\frac{1}{4} & - \frac{\sqrt{2} i}{4} & - \frac{1}{4}\\\frac{\sqrt{2} i}{4} & \frac{1}{2} & - \frac{\sqrt{2} i}{4}\\- \frac{1}{4} & \frac{\sqrt{2} i}{4} & \frac{1}{4}\end{matrix}\right]$$

    
    
    \begin{verbatim}
'Projector?'
    \end{verbatim}

    
    
    \begin{verbatim}
True
    \end{verbatim}

    
    $$\left[\begin{matrix}\frac{1}{4} & - \frac{\sqrt{2} i}{4} & - \frac{1}{4}\\\frac{\sqrt{2} i}{4} & \frac{1}{2} & - \frac{\sqrt{2} i}{4}\\- \frac{1}{4} & \frac{\sqrt{2} i}{4} & \frac{1}{4}\end{matrix}\right]$$

    
    $$\left[\begin{matrix}\frac{1}{4} & - \frac{\sqrt{2} i}{4} & - \frac{1}{4}\\\frac{\sqrt{2} i}{4} & \frac{1}{2} & - \frac{\sqrt{2} i}{4}\\- \frac{1}{4} & \frac{\sqrt{2} i}{4} & \frac{1}{4}\end{matrix}\right]$$

    
    
    \begin{verbatim}
'Closed?'
    \end{verbatim}

    
    
    \begin{verbatim}
True
    \end{verbatim}

    
    
    \begin{verbatim}
'Orthogonal?'
    \end{verbatim}

    
    
    \begin{verbatim}
True
    \end{verbatim}

    
    \hypertarget{section}{%
\subsubsection{3.}\label{section}}

Para los kets dados, debemos comprobar
\[\langle \psi_i\lvert\psi_i\rangle = 1\]

Comencemos con \(\lvert\psi_0\rangle\):

\[\langle \psi_0\lvert\psi_0\rangle = \frac{1}{2}+\frac{1}{4}+\frac{1}{4} = 1,\]
entonces, podemos concluir que este ket está normalizado. Note que los
únicos términos sobrevivientes del bracket son los
\(\langle u_i\lvert u_i\rangle\), ya que los demás son cero por
ortogonalidad. Ahora,

\[\langle \psi_1\lvert\psi_1\rangle = \frac{1}{3} + \frac{1}{3} = \frac{2}{3} \neq 1,\]
por lo tanto, el segundo ket no está normalizado. Al normalizarlo,
tenemos que
\[\lvert\psi_1\rangle = \frac{1}{\sqrt{2}}\lvert u_1\rangle + \frac{i}{\sqrt{2}}\lvert u_3\rangle.\]

Luego, debemos definir los operadores
\[\rho_i = \lvert\psi_i\rangle\langle\psi_i\rvert\]

    \begin{Verbatim}[commandchars=\\\{\}]
{\color{incolor}In [{\color{incolor}9}]:} \PY{n}{psi\PYZus{}0} \PY{o}{=} \PY{n}{Matrix}\PY{p}{(}\PY{p}{[}\PY{p}{[}\PY{l+m+mi}{1}\PY{o}{/}\PY{n}{sqrt}\PY{p}{(}\PY{l+m+mi}{2}\PY{p}{)}\PY{p}{]}\PY{p}{,}\PY{p}{[}\PY{n}{I}\PY{o}{/}\PY{l+m+mi}{2}\PY{p}{]}\PY{p}{,}\PY{p}{[}\PY{n}{nsimplify}\PY{p}{(}\PY{l+m+mf}{1.}\PY{o}{/}\PY{l+m+mi}{2}\PY{p}{)}\PY{p}{]}\PY{p}{]}\PY{p}{)}
        \PY{n}{rho\PYZus{}0} \PY{o}{=} \PY{n}{psi\PYZus{}0}\PY{o}{*}\PY{n}{psi\PYZus{}0}\PY{o}{.}\PY{n}{transpose}\PY{p}{(}\PY{p}{)}\PY{o}{.}\PY{n}{conjugate}\PY{p}{(}\PY{p}{)}
        \PY{n}{display}\PY{p}{(}\PY{l+s+s1}{\PYZsq{}}\PY{l+s+s1}{rho\PYZus{}0 = }\PY{l+s+s1}{\PYZsq{}}\PY{p}{,}\PY{n}{rho\PYZus{}0}\PY{p}{)}
        \PY{n}{display}\PY{p}{(}\PY{l+s+s1}{\PYZsq{}}\PY{l+s+s1}{adj(rho\PYZus{}0) = }\PY{l+s+s1}{\PYZsq{}}\PY{p}{,}\PY{n}{rho\PYZus{}0}\PY{o}{.}\PY{n}{transpose}\PY{p}{(}\PY{p}{)}\PY{o}{.}\PY{n}{conjugate}\PY{p}{(}\PY{p}{)}\PY{p}{)}
        \PY{n}{display}\PY{p}{(}\PY{l+s+s1}{\PYZsq{}}\PY{l+s+s1}{Hermitian?}\PY{l+s+s1}{\PYZsq{}}\PY{p}{,}\PY{n}{rho\PYZus{}0}\PY{o}{.}\PY{n}{transpose}\PY{p}{(}\PY{p}{)}\PY{o}{.}\PY{n}{conjugate}\PY{p}{(}\PY{p}{)}\PY{o}{==}\PY{n}{rho\PYZus{}0}\PY{p}{)}
\end{Verbatim}


    
    \begin{verbatim}
'rho_0 = '
    \end{verbatim}

    
    $$\left[\begin{matrix}\frac{1}{2} & - \frac{\sqrt{2} i}{4} & \frac{\sqrt{2}}{4}\\\frac{\sqrt{2} i}{4} & \frac{1}{4} & \frac{i}{4}\\\frac{\sqrt{2}}{4} & - \frac{i}{4} & \frac{1}{4}\end{matrix}\right]$$

    
    
    \begin{verbatim}
'adj(rho_0) = '
    \end{verbatim}

    
    $$\left[\begin{matrix}\frac{1}{2} & - \frac{\sqrt{2} i}{4} & \frac{\sqrt{2}}{4}\\\frac{\sqrt{2} i}{4} & \frac{1}{4} & \frac{i}{4}\\\frac{\sqrt{2}}{4} & - \frac{i}{4} & \frac{1}{4}\end{matrix}\right]$$

    
    
    \begin{verbatim}
'Hermitian?'
    \end{verbatim}

    
    
    \begin{verbatim}
True
    \end{verbatim}

    
    \begin{Verbatim}[commandchars=\\\{\}]
{\color{incolor}In [{\color{incolor}10}]:} \PY{n}{psi\PYZus{}1} \PY{o}{=} \PY{n}{Matrix}\PY{p}{(}\PY{p}{[}\PY{p}{[}\PY{l+m+mi}{1}\PY{o}{/}\PY{n}{sqrt}\PY{p}{(}\PY{l+m+mi}{2}\PY{p}{)}\PY{p}{]}\PY{p}{,}\PY{p}{[}\PY{l+m+mi}{0}\PY{p}{]}\PY{p}{,}\PY{p}{[}\PY{l+m+mi}{1}\PY{o}{/}\PY{n}{sqrt}\PY{p}{(}\PY{l+m+mi}{2}\PY{p}{)}\PY{p}{]}\PY{p}{]}\PY{p}{)}
         \PY{n}{rho\PYZus{}1} \PY{o}{=} \PY{n}{psi\PYZus{}1}\PY{o}{*}\PY{n}{psi\PYZus{}1}\PY{o}{.}\PY{n}{transpose}\PY{p}{(}\PY{p}{)}\PY{o}{.}\PY{n}{conjugate}\PY{p}{(}\PY{p}{)}
         \PY{n}{display}\PY{p}{(}\PY{l+s+s1}{\PYZsq{}}\PY{l+s+s1}{rho\PYZus{}1 = }\PY{l+s+s1}{\PYZsq{}}\PY{p}{,}\PY{n}{rho\PYZus{}1}\PY{p}{)}
         \PY{n}{display}\PY{p}{(}\PY{l+s+s1}{\PYZsq{}}\PY{l+s+s1}{adj(rho\PYZus{}1) = }\PY{l+s+s1}{\PYZsq{}}\PY{p}{,}\PY{n}{rho\PYZus{}1}\PY{o}{.}\PY{n}{transpose}\PY{p}{(}\PY{p}{)}\PY{o}{.}\PY{n}{conjugate}\PY{p}{(}\PY{p}{)}\PY{p}{)}
         \PY{n}{display}\PY{p}{(}\PY{l+s+s1}{\PYZsq{}}\PY{l+s+s1}{Hermitian?}\PY{l+s+s1}{\PYZsq{}}\PY{p}{,}\PY{n}{rho\PYZus{}1}\PY{o}{.}\PY{n}{transpose}\PY{p}{(}\PY{p}{)}\PY{o}{.}\PY{n}{conjugate}\PY{p}{(}\PY{p}{)}\PY{o}{==}\PY{n}{rho\PYZus{}1}\PY{p}{)}
\end{Verbatim}


    
    \begin{verbatim}
'rho_1 = '
    \end{verbatim}

    
    $$\left[\begin{matrix}\frac{1}{2} & 0 & \frac{1}{2}\\0 & 0 & 0\\\frac{1}{2} & 0 & \frac{1}{2}\end{matrix}\right]$$

    
    
    \begin{verbatim}
'adj(rho_1) = '
    \end{verbatim}

    
    $$\left[\begin{matrix}\frac{1}{2} & 0 & \frac{1}{2}\\0 & 0 & 0\\\frac{1}{2} & 0 & \frac{1}{2}\end{matrix}\right]$$

    
    
    \begin{verbatim}
'Hermitian?'
    \end{verbatim}

    
    
    \begin{verbatim}
True
    \end{verbatim}

    
    \hypertarget{section}{%
\subsubsection{5.}\label{section}}

Finalmente, tenemos que dados \(P_i\) ortogonales, es decir
\(P_i = P_i^\dagger\) (autoadjuntas), una proyección \(P_1P_2\) es
ortogonal, si
\[P_1P_2 = (P_1P_2)^\dagger = P_2^\dagger P_1^\dagger = P_2P_1,\] es
decir, si \[[P_1,P_2]=0.\]

Sea \(\mathcal{V}\) el subespacio al cual proyecta \(P_1P_2\), es decir,
\(P_1P_2\lvert\alpha\rangle=\lvert\alpha\rangle\) implica que
\(\lvert\alpha\rangle \in \mathcal{V}\). Ahora, si aplicamos
\(P_1\lvert\alpha\rangle = P_1^2P_2\lvert\alpha\rangle=P_1P_2\lvert\alpha\rangle = \lvert\alpha\rangle\)
nos damos cuenta que \(\lvert\alpha\rangle\in \mathcal{S}_1\). Si
repetimos el proceso con \(P_2\), tenemos que
\(P_2\lvert\alpha\rangle=P_2P_1P_2\lvert\alpha\rangle = P_2^2P_1 \lvert\alpha\rangle = P_2P_1\lvert\alpha\rangle=P_1P_2\lvert\alpha\rangle=\lvert\alpha\rangle\)
es evidente que \(\lvert\alpha\rangle \in \mathcal{S}_2\). De todo lo
anterior, podemos concluir que si \(\lvert\alpha\rangle\) está tanto en
\(\mathcal{S}_1\) como en \(\mathcal{S}_2\), el subespacio al que
proyecta el producto debe ser
\(\mathcal{V} = \mathcal{S}_1\cap\mathcal{S}_2\).


    % Add a bibliography block to the postdoc
    
    
    
    \end{document}
