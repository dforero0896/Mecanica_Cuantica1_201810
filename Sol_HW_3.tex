
% Default to the notebook output style

    


% Inherit from the specified cell style.




    
\documentclass[11pt]{article}

    
    
    \usepackage[T1]{fontenc}
    % Nicer default font than Computer Modern for most use cases
    \usepackage{palatino}

    % Basic figure setup, for now with no caption control since it's done
    % automatically by Pandoc (which extracts ![](path) syntax from Markdown).
    \usepackage{graphicx}
    % We will generate all images so they have a width \maxwidth. This means
    % that they will get their normal width if they fit onto the page, but
    % are scaled down if they would overflow the margins.
    \makeatletter
    \def\maxwidth{\ifdim\Gin@nat@width>\linewidth\linewidth
    \else\Gin@nat@width\fi}
    \makeatother
    \let\Oldincludegraphics\includegraphics
    % Set max figure width to be 80% of text width, for now hardcoded.
    \renewcommand{\includegraphics}[1]{\Oldincludegraphics[width=.8\maxwidth]{#1}}
    % Ensure that by default, figures have no caption (until we provide a
    % proper Figure object with a Caption API and a way to capture that
    % in the conversion process - todo).
    \usepackage{caption}
    \DeclareCaptionLabelFormat{nolabel}{}
    \captionsetup{labelformat=nolabel}

    \usepackage{adjustbox} % Used to constrain images to a maximum size 
    \usepackage{xcolor} % Allow colors to be defined
    \usepackage{enumerate} % Needed for markdown enumerations to work
    \usepackage{geometry} % Used to adjust the document margins
    \usepackage{amsmath} % Equations
    \usepackage{amssymb} % Equations
    \usepackage{textcomp} % defines textquotesingle
    % Hack from http://tex.stackexchange.com/a/47451/13684:
    \AtBeginDocument{%
        \def\PYZsq{\textquotesingle}% Upright quotes in Pygmentized code
    }
    \usepackage{upquote} % Upright quotes for verbatim code
    \usepackage{eurosym} % defines \euro
    \usepackage[mathletters]{ucs} % Extended unicode (utf-8) support
    \usepackage[utf8x]{inputenc} % Allow utf-8 characters in the tex document
    \usepackage{fancyvrb} % verbatim replacement that allows latex
    \usepackage{grffile} % extends the file name processing of package graphics 
                         % to support a larger range 
    % The hyperref package gives us a pdf with properly built
    % internal navigation ('pdf bookmarks' for the table of contents,
    % internal cross-reference links, web links for URLs, etc.)
    \usepackage{hyperref}
    \usepackage{longtable} % longtable support required by pandoc >1.10
    \usepackage{booktabs}  % table support for pandoc > 1.12.2
    \usepackage[normalem]{ulem} % ulem is needed to support strikethroughs (\sout)
                                % normalem makes italics be italics, not underlines
    

    
    
    % Colors for the hyperref package
    \definecolor{urlcolor}{rgb}{0,.145,.698}
    \definecolor{linkcolor}{rgb}{.71,0.21,0.01}
    \definecolor{citecolor}{rgb}{.12,.54,.11}

    % ANSI colors
    \definecolor{ansi-black}{HTML}{3E424D}
    \definecolor{ansi-black-intense}{HTML}{282C36}
    \definecolor{ansi-red}{HTML}{E75C58}
    \definecolor{ansi-red-intense}{HTML}{B22B31}
    \definecolor{ansi-green}{HTML}{00A250}
    \definecolor{ansi-green-intense}{HTML}{007427}
    \definecolor{ansi-yellow}{HTML}{DDB62B}
    \definecolor{ansi-yellow-intense}{HTML}{B27D12}
    \definecolor{ansi-blue}{HTML}{208FFB}
    \definecolor{ansi-blue-intense}{HTML}{0065CA}
    \definecolor{ansi-magenta}{HTML}{D160C4}
    \definecolor{ansi-magenta-intense}{HTML}{A03196}
    \definecolor{ansi-cyan}{HTML}{60C6C8}
    \definecolor{ansi-cyan-intense}{HTML}{258F8F}
    \definecolor{ansi-white}{HTML}{C5C1B4}
    \definecolor{ansi-white-intense}{HTML}{A1A6B2}

    % commands and environments needed by pandoc snippets
    % extracted from the output of `pandoc -s`
    \providecommand{\tightlist}{%
      \setlength{\itemsep}{0pt}\setlength{\parskip}{0pt}}
    \DefineVerbatimEnvironment{Highlighting}{Verbatim}{commandchars=\\\{\}}
    % Add ',fontsize=\small' for more characters per line
    \newenvironment{Shaded}{}{}
    \newcommand{\KeywordTok}[1]{\textcolor[rgb]{0.00,0.44,0.13}{\textbf{{#1}}}}
    \newcommand{\DataTypeTok}[1]{\textcolor[rgb]{0.56,0.13,0.00}{{#1}}}
    \newcommand{\DecValTok}[1]{\textcolor[rgb]{0.25,0.63,0.44}{{#1}}}
    \newcommand{\BaseNTok}[1]{\textcolor[rgb]{0.25,0.63,0.44}{{#1}}}
    \newcommand{\FloatTok}[1]{\textcolor[rgb]{0.25,0.63,0.44}{{#1}}}
    \newcommand{\CharTok}[1]{\textcolor[rgb]{0.25,0.44,0.63}{{#1}}}
    \newcommand{\StringTok}[1]{\textcolor[rgb]{0.25,0.44,0.63}{{#1}}}
    \newcommand{\CommentTok}[1]{\textcolor[rgb]{0.38,0.63,0.69}{\textit{{#1}}}}
    \newcommand{\OtherTok}[1]{\textcolor[rgb]{0.00,0.44,0.13}{{#1}}}
    \newcommand{\AlertTok}[1]{\textcolor[rgb]{1.00,0.00,0.00}{\textbf{{#1}}}}
    \newcommand{\FunctionTok}[1]{\textcolor[rgb]{0.02,0.16,0.49}{{#1}}}
    \newcommand{\RegionMarkerTok}[1]{{#1}}
    \newcommand{\ErrorTok}[1]{\textcolor[rgb]{1.00,0.00,0.00}{\textbf{{#1}}}}
    \newcommand{\NormalTok}[1]{{#1}}
    
    % Additional commands for more recent versions of Pandoc
    \newcommand{\ConstantTok}[1]{\textcolor[rgb]{0.53,0.00,0.00}{{#1}}}
    \newcommand{\SpecialCharTok}[1]{\textcolor[rgb]{0.25,0.44,0.63}{{#1}}}
    \newcommand{\VerbatimStringTok}[1]{\textcolor[rgb]{0.25,0.44,0.63}{{#1}}}
    \newcommand{\SpecialStringTok}[1]{\textcolor[rgb]{0.73,0.40,0.53}{{#1}}}
    \newcommand{\ImportTok}[1]{{#1}}
    \newcommand{\DocumentationTok}[1]{\textcolor[rgb]{0.73,0.13,0.13}{\textit{{#1}}}}
    \newcommand{\AnnotationTok}[1]{\textcolor[rgb]{0.38,0.63,0.69}{\textbf{\textit{{#1}}}}}
    \newcommand{\CommentVarTok}[1]{\textcolor[rgb]{0.38,0.63,0.69}{\textbf{\textit{{#1}}}}}
    \newcommand{\VariableTok}[1]{\textcolor[rgb]{0.10,0.09,0.49}{{#1}}}
    \newcommand{\ControlFlowTok}[1]{\textcolor[rgb]{0.00,0.44,0.13}{\textbf{{#1}}}}
    \newcommand{\OperatorTok}[1]{\textcolor[rgb]{0.40,0.40,0.40}{{#1}}}
    \newcommand{\BuiltInTok}[1]{{#1}}
    \newcommand{\ExtensionTok}[1]{{#1}}
    \newcommand{\PreprocessorTok}[1]{\textcolor[rgb]{0.74,0.48,0.00}{{#1}}}
    \newcommand{\AttributeTok}[1]{\textcolor[rgb]{0.49,0.56,0.16}{{#1}}}
    \newcommand{\InformationTok}[1]{\textcolor[rgb]{0.38,0.63,0.69}{\textbf{\textit{{#1}}}}}
    \newcommand{\WarningTok}[1]{\textcolor[rgb]{0.38,0.63,0.69}{\textbf{\textit{{#1}}}}}
    
    
    % Define a nice break command that doesn't care if a line doesn't already
    % exist.
    \def\br{\hspace*{\fill} \\* }
    % Math Jax compatability definitions
    \def\gt{>}
    \def\lt{<}
    % Document parameters
    \title{Sol\_HW\_3}
    
    
    

    % Pygments definitions
    
\makeatletter
\def\PY@reset{\let\PY@it=\relax \let\PY@bf=\relax%
    \let\PY@ul=\relax \let\PY@tc=\relax%
    \let\PY@bc=\relax \let\PY@ff=\relax}
\def\PY@tok#1{\csname PY@tok@#1\endcsname}
\def\PY@toks#1+{\ifx\relax#1\empty\else%
    \PY@tok{#1}\expandafter\PY@toks\fi}
\def\PY@do#1{\PY@bc{\PY@tc{\PY@ul{%
    \PY@it{\PY@bf{\PY@ff{#1}}}}}}}
\def\PY#1#2{\PY@reset\PY@toks#1+\relax+\PY@do{#2}}

\expandafter\def\csname PY@tok@gd\endcsname{\def\PY@tc##1{\textcolor[rgb]{0.63,0.00,0.00}{##1}}}
\expandafter\def\csname PY@tok@gu\endcsname{\let\PY@bf=\textbf\def\PY@tc##1{\textcolor[rgb]{0.50,0.00,0.50}{##1}}}
\expandafter\def\csname PY@tok@gt\endcsname{\def\PY@tc##1{\textcolor[rgb]{0.00,0.27,0.87}{##1}}}
\expandafter\def\csname PY@tok@gs\endcsname{\let\PY@bf=\textbf}
\expandafter\def\csname PY@tok@gr\endcsname{\def\PY@tc##1{\textcolor[rgb]{1.00,0.00,0.00}{##1}}}
\expandafter\def\csname PY@tok@cm\endcsname{\let\PY@it=\textit\def\PY@tc##1{\textcolor[rgb]{0.25,0.50,0.50}{##1}}}
\expandafter\def\csname PY@tok@vg\endcsname{\def\PY@tc##1{\textcolor[rgb]{0.10,0.09,0.49}{##1}}}
\expandafter\def\csname PY@tok@vi\endcsname{\def\PY@tc##1{\textcolor[rgb]{0.10,0.09,0.49}{##1}}}
\expandafter\def\csname PY@tok@mh\endcsname{\def\PY@tc##1{\textcolor[rgb]{0.40,0.40,0.40}{##1}}}
\expandafter\def\csname PY@tok@cs\endcsname{\let\PY@it=\textit\def\PY@tc##1{\textcolor[rgb]{0.25,0.50,0.50}{##1}}}
\expandafter\def\csname PY@tok@ge\endcsname{\let\PY@it=\textit}
\expandafter\def\csname PY@tok@vc\endcsname{\def\PY@tc##1{\textcolor[rgb]{0.10,0.09,0.49}{##1}}}
\expandafter\def\csname PY@tok@il\endcsname{\def\PY@tc##1{\textcolor[rgb]{0.40,0.40,0.40}{##1}}}
\expandafter\def\csname PY@tok@go\endcsname{\def\PY@tc##1{\textcolor[rgb]{0.53,0.53,0.53}{##1}}}
\expandafter\def\csname PY@tok@cp\endcsname{\def\PY@tc##1{\textcolor[rgb]{0.74,0.48,0.00}{##1}}}
\expandafter\def\csname PY@tok@gi\endcsname{\def\PY@tc##1{\textcolor[rgb]{0.00,0.63,0.00}{##1}}}
\expandafter\def\csname PY@tok@gh\endcsname{\let\PY@bf=\textbf\def\PY@tc##1{\textcolor[rgb]{0.00,0.00,0.50}{##1}}}
\expandafter\def\csname PY@tok@ni\endcsname{\let\PY@bf=\textbf\def\PY@tc##1{\textcolor[rgb]{0.60,0.60,0.60}{##1}}}
\expandafter\def\csname PY@tok@nl\endcsname{\def\PY@tc##1{\textcolor[rgb]{0.63,0.63,0.00}{##1}}}
\expandafter\def\csname PY@tok@nn\endcsname{\let\PY@bf=\textbf\def\PY@tc##1{\textcolor[rgb]{0.00,0.00,1.00}{##1}}}
\expandafter\def\csname PY@tok@no\endcsname{\def\PY@tc##1{\textcolor[rgb]{0.53,0.00,0.00}{##1}}}
\expandafter\def\csname PY@tok@na\endcsname{\def\PY@tc##1{\textcolor[rgb]{0.49,0.56,0.16}{##1}}}
\expandafter\def\csname PY@tok@nb\endcsname{\def\PY@tc##1{\textcolor[rgb]{0.00,0.50,0.00}{##1}}}
\expandafter\def\csname PY@tok@nc\endcsname{\let\PY@bf=\textbf\def\PY@tc##1{\textcolor[rgb]{0.00,0.00,1.00}{##1}}}
\expandafter\def\csname PY@tok@nd\endcsname{\def\PY@tc##1{\textcolor[rgb]{0.67,0.13,1.00}{##1}}}
\expandafter\def\csname PY@tok@ne\endcsname{\let\PY@bf=\textbf\def\PY@tc##1{\textcolor[rgb]{0.82,0.25,0.23}{##1}}}
\expandafter\def\csname PY@tok@nf\endcsname{\def\PY@tc##1{\textcolor[rgb]{0.00,0.00,1.00}{##1}}}
\expandafter\def\csname PY@tok@si\endcsname{\let\PY@bf=\textbf\def\PY@tc##1{\textcolor[rgb]{0.73,0.40,0.53}{##1}}}
\expandafter\def\csname PY@tok@s2\endcsname{\def\PY@tc##1{\textcolor[rgb]{0.73,0.13,0.13}{##1}}}
\expandafter\def\csname PY@tok@nt\endcsname{\let\PY@bf=\textbf\def\PY@tc##1{\textcolor[rgb]{0.00,0.50,0.00}{##1}}}
\expandafter\def\csname PY@tok@nv\endcsname{\def\PY@tc##1{\textcolor[rgb]{0.10,0.09,0.49}{##1}}}
\expandafter\def\csname PY@tok@s1\endcsname{\def\PY@tc##1{\textcolor[rgb]{0.73,0.13,0.13}{##1}}}
\expandafter\def\csname PY@tok@ch\endcsname{\let\PY@it=\textit\def\PY@tc##1{\textcolor[rgb]{0.25,0.50,0.50}{##1}}}
\expandafter\def\csname PY@tok@m\endcsname{\def\PY@tc##1{\textcolor[rgb]{0.40,0.40,0.40}{##1}}}
\expandafter\def\csname PY@tok@gp\endcsname{\let\PY@bf=\textbf\def\PY@tc##1{\textcolor[rgb]{0.00,0.00,0.50}{##1}}}
\expandafter\def\csname PY@tok@sh\endcsname{\def\PY@tc##1{\textcolor[rgb]{0.73,0.13,0.13}{##1}}}
\expandafter\def\csname PY@tok@ow\endcsname{\let\PY@bf=\textbf\def\PY@tc##1{\textcolor[rgb]{0.67,0.13,1.00}{##1}}}
\expandafter\def\csname PY@tok@sx\endcsname{\def\PY@tc##1{\textcolor[rgb]{0.00,0.50,0.00}{##1}}}
\expandafter\def\csname PY@tok@bp\endcsname{\def\PY@tc##1{\textcolor[rgb]{0.00,0.50,0.00}{##1}}}
\expandafter\def\csname PY@tok@c1\endcsname{\let\PY@it=\textit\def\PY@tc##1{\textcolor[rgb]{0.25,0.50,0.50}{##1}}}
\expandafter\def\csname PY@tok@o\endcsname{\def\PY@tc##1{\textcolor[rgb]{0.40,0.40,0.40}{##1}}}
\expandafter\def\csname PY@tok@kc\endcsname{\let\PY@bf=\textbf\def\PY@tc##1{\textcolor[rgb]{0.00,0.50,0.00}{##1}}}
\expandafter\def\csname PY@tok@c\endcsname{\let\PY@it=\textit\def\PY@tc##1{\textcolor[rgb]{0.25,0.50,0.50}{##1}}}
\expandafter\def\csname PY@tok@mf\endcsname{\def\PY@tc##1{\textcolor[rgb]{0.40,0.40,0.40}{##1}}}
\expandafter\def\csname PY@tok@err\endcsname{\def\PY@bc##1{\setlength{\fboxsep}{0pt}\fcolorbox[rgb]{1.00,0.00,0.00}{1,1,1}{\strut ##1}}}
\expandafter\def\csname PY@tok@mb\endcsname{\def\PY@tc##1{\textcolor[rgb]{0.40,0.40,0.40}{##1}}}
\expandafter\def\csname PY@tok@ss\endcsname{\def\PY@tc##1{\textcolor[rgb]{0.10,0.09,0.49}{##1}}}
\expandafter\def\csname PY@tok@sr\endcsname{\def\PY@tc##1{\textcolor[rgb]{0.73,0.40,0.53}{##1}}}
\expandafter\def\csname PY@tok@mo\endcsname{\def\PY@tc##1{\textcolor[rgb]{0.40,0.40,0.40}{##1}}}
\expandafter\def\csname PY@tok@kd\endcsname{\let\PY@bf=\textbf\def\PY@tc##1{\textcolor[rgb]{0.00,0.50,0.00}{##1}}}
\expandafter\def\csname PY@tok@mi\endcsname{\def\PY@tc##1{\textcolor[rgb]{0.40,0.40,0.40}{##1}}}
\expandafter\def\csname PY@tok@kn\endcsname{\let\PY@bf=\textbf\def\PY@tc##1{\textcolor[rgb]{0.00,0.50,0.00}{##1}}}
\expandafter\def\csname PY@tok@cpf\endcsname{\let\PY@it=\textit\def\PY@tc##1{\textcolor[rgb]{0.25,0.50,0.50}{##1}}}
\expandafter\def\csname PY@tok@kr\endcsname{\let\PY@bf=\textbf\def\PY@tc##1{\textcolor[rgb]{0.00,0.50,0.00}{##1}}}
\expandafter\def\csname PY@tok@s\endcsname{\def\PY@tc##1{\textcolor[rgb]{0.73,0.13,0.13}{##1}}}
\expandafter\def\csname PY@tok@kp\endcsname{\def\PY@tc##1{\textcolor[rgb]{0.00,0.50,0.00}{##1}}}
\expandafter\def\csname PY@tok@w\endcsname{\def\PY@tc##1{\textcolor[rgb]{0.73,0.73,0.73}{##1}}}
\expandafter\def\csname PY@tok@kt\endcsname{\def\PY@tc##1{\textcolor[rgb]{0.69,0.00,0.25}{##1}}}
\expandafter\def\csname PY@tok@sc\endcsname{\def\PY@tc##1{\textcolor[rgb]{0.73,0.13,0.13}{##1}}}
\expandafter\def\csname PY@tok@sb\endcsname{\def\PY@tc##1{\textcolor[rgb]{0.73,0.13,0.13}{##1}}}
\expandafter\def\csname PY@tok@k\endcsname{\let\PY@bf=\textbf\def\PY@tc##1{\textcolor[rgb]{0.00,0.50,0.00}{##1}}}
\expandafter\def\csname PY@tok@se\endcsname{\let\PY@bf=\textbf\def\PY@tc##1{\textcolor[rgb]{0.73,0.40,0.13}{##1}}}
\expandafter\def\csname PY@tok@sd\endcsname{\let\PY@it=\textit\def\PY@tc##1{\textcolor[rgb]{0.73,0.13,0.13}{##1}}}

\def\PYZbs{\char`\\}
\def\PYZus{\char`\_}
\def\PYZob{\char`\{}
\def\PYZcb{\char`\}}
\def\PYZca{\char`\^}
\def\PYZam{\char`\&}
\def\PYZlt{\char`\<}
\def\PYZgt{\char`\>}
\def\PYZsh{\char`\#}
\def\PYZpc{\char`\%}
\def\PYZdl{\char`\$}
\def\PYZhy{\char`\-}
\def\PYZsq{\char`\'}
\def\PYZdq{\char`\"}
\def\PYZti{\char`\~}
% for compatibility with earlier versions
\def\PYZat{@}
\def\PYZlb{[}
\def\PYZrb{]}
\makeatother


    % Exact colors from NB
    \definecolor{incolor}{rgb}{0.0, 0.0, 0.5}
    \definecolor{outcolor}{rgb}{0.545, 0.0, 0.0}



    
    % Prevent overflowing lines due to hard-to-break entities
    \sloppy 
    % Setup hyperref package
    \hypersetup{
      breaklinks=true,  % so long urls are correctly broken across lines
      colorlinks=true,
      urlcolor=urlcolor,
      linkcolor=linkcolor,
      citecolor=citecolor,
      }
    % Slightly bigger margins than the latex defaults
    
    \geometry{verbose,tmargin=1in,bmargin=1in,lmargin=1in,rmargin=1in}
    
    

    \begin{document}
    
    
    \maketitle
    
    

    
    \section{Mecánica Cuántica 1 -
201820}\label{mecuxe1nica-cuuxe1ntica-1---201820}

\subsection{Tarea \# 3 - Solución}\label{tarea-3---soluciuxf3n}

Elaborada por Daniel Forero

    \begin{Verbatim}[commandchars=\\\{\}]
{\color{incolor}In [{\color{incolor}1}]:} \PY{k+kn}{from} \PY{n+nn}{sympy} \PY{k+kn}{import} \PY{o}{*}
        \PY{n}{init\PYZus{}printing}\PY{p}{(}\PY{n}{use\PYZus{}latex}\PY{o}{=}\PY{l+s+s1}{\PYZsq{}}\PY{l+s+s1}{mathjax}\PY{l+s+s1}{\PYZsq{}}\PY{p}{)}
        \PY{k+kn}{from} \PY{n+nn}{IPython.display} \PY{k+kn}{import} \PY{n}{display}
\end{Verbatim}

    \subsection{Problema 3.4}\label{problema-3.4}

    Nuestra función inicial \(\psi(x)\) tiene \(\Delta p = \sigma\) y
\(\langle p \rangle = q\). Si aplicamos una transformación
\[\phi(x)=\exp\left(\frac{ip_0x}{\hbar}\right) \psi(x).\ \ \ p_0 = \hbar k_0\]
Esta transformada de Fourier se escribiría
\[\hat{\phi}(k) = \int_{-\infty}^{\infty}\exp(ik_0x) \psi(x)\exp(-ikx)dk,\]

\[\hat{\phi}(k) = \int_{-\infty}^{\infty}\psi(x)\exp(-i(k-k_0)x)dk,\ \ \ k-k_0 = k'\]

\[\hat{\phi}(k'+k_0) = \int_{-\infty}^{\infty}\psi(x)\exp(-ik'x)dk'\]

De esta forma vemos que la transformación en el espacio es equivalente a
una translación en espacio de momentum. Esta translación, claramente,
afecta el valor esperado, ya que la distribución se encuentra ahora
centrada en \(q+p_0\). Este será el valor esperado.

Por otro lado, la translación no modifica la forma de la distribución,
de manera que \(\Delta p = \sigma\).

    \subsection{Problema 3.6}\label{problema-3.6}

    \begin{Verbatim}[commandchars=\\\{\}]
{\color{incolor}In [{\color{incolor}2}]:} \PY{n}{A}\PY{p}{,}\PY{n}{B}\PY{p}{,}\PY{n}{C}\PY{p}{,}\PY{n}{D} \PY{o}{=} \PY{n}{symbols}\PY{p}{(}\PY{l+s+s1}{\PYZsq{}}\PY{l+s+s1}{A B C D}\PY{l+s+s1}{\PYZsq{}}\PY{p}{)}
        \PY{n}{q}\PY{p}{,} \PY{n}{k}\PY{p}{,} \PY{n}{x}\PY{p}{,} \PY{n}{a} \PY{o}{=} \PY{n}{symbols}\PY{p}{(}\PY{l+s+s1}{\PYZsq{}}\PY{l+s+s1}{q k x a}\PY{l+s+s1}{\PYZsq{}}\PY{p}{,} \PY{n}{real}\PY{o}{=}\PY{n+nb+bp}{True}\PY{p}{)}
        \PY{n}{psi\PYZus{}1} \PY{o}{=} \PY{n}{exp}\PY{p}{(}\PY{n}{I}\PY{o}{*}\PY{n}{k}\PY{o}{*}\PY{n}{x}\PY{p}{)} \PY{o}{+} \PY{n}{A}\PY{o}{*}\PY{n}{exp}\PY{p}{(}\PY{o}{\PYZhy{}}\PY{n}{I}\PY{o}{*}\PY{n}{k}\PY{o}{*}\PY{n}{x}\PY{p}{)}
        \PY{n}{psi\PYZus{}2} \PY{o}{=} \PY{n}{B}\PY{o}{*}\PY{n}{exp}\PY{p}{(}\PY{n}{I}\PY{o}{*}\PY{n}{q}\PY{o}{*}\PY{n}{x}\PY{p}{)} \PY{o}{+} \PY{n}{C}\PY{o}{*}\PY{n}{exp}\PY{p}{(}\PY{o}{\PYZhy{}}\PY{n}{I}\PY{o}{*}\PY{n}{q}\PY{o}{*}\PY{n}{x}\PY{p}{)}
        \PY{n}{psi\PYZus{}3} \PY{o}{=} \PY{n}{D}\PY{o}{*}\PY{n}{exp}\PY{p}{(}\PY{n}{I}\PY{o}{*}\PY{n}{k}\PY{o}{*}\PY{n}{x}\PY{p}{)}
        \PY{c+c1}{\PYZsh{} En x = \PYZhy{}a}
        \PY{n}{step\PYZus{}1}\PY{o}{=}\PY{n}{simplify}\PY{p}{(}\PY{n}{psi\PYZus{}1}\PY{o}{.}\PY{n}{subs}\PY{p}{(}\PY{n}{x}\PY{p}{,}\PY{o}{\PYZhy{}}\PY{n}{a}\PY{p}{)}\PY{o}{\PYZhy{}}\PY{n}{psi\PYZus{}2}\PY{o}{.}\PY{n}{subs}\PY{p}{(}\PY{n}{x}\PY{p}{,}\PY{o}{\PYZhy{}}\PY{n}{a}\PY{p}{)}\PY{p}{)}
        \PY{n}{display}\PY{p}{(}\PY{l+s+s1}{\PYZsq{}}\PY{l+s+s1}{En x= \PYZhy{}a}\PY{l+s+s1}{\PYZsq{}}\PY{p}{,}\PY{n}{Eq}\PY{p}{(}\PY{n}{step\PYZus{}1}\PY{p}{,}\PY{l+m+mi}{0}\PY{p}{)}\PY{p}{)}
        \PY{n}{step\PYZus{}1d}\PY{o}{=}\PY{n}{simplify}\PY{p}{(}\PY{n}{diff}\PY{p}{(}\PY{n}{psi\PYZus{}1}\PY{p}{,}\PY{n}{x}\PY{p}{)}\PY{o}{.}\PY{n}{subs}\PY{p}{(}\PY{n}{x}\PY{p}{,}\PY{o}{\PYZhy{}}\PY{n}{a}\PY{p}{)}\PYZbs{}
                         \PY{o}{\PYZhy{}}\PY{n}{diff}\PY{p}{(}\PY{n}{psi\PYZus{}2}\PY{p}{,}\PY{n}{x}\PY{p}{)}\PY{o}{.}\PY{n}{subs}\PY{p}{(}\PY{n}{x}\PY{p}{,}\PY{o}{\PYZhy{}}\PY{n}{a}\PY{p}{)}\PY{p}{)}
        \PY{n}{display}\PY{p}{(}\PY{l+s+s1}{\PYZsq{}}\PY{l+s+s1}{En x= \PYZhy{}a}\PY{l+s+s1}{\PYZsq{}}\PY{p}{,}\PY{n}{Eq}\PY{p}{(}\PY{n}{step\PYZus{}1d}\PY{p}{,}\PY{l+m+mi}{0}\PY{p}{)}\PY{p}{)}
        \PY{c+c1}{\PYZsh{} En x = a}
        \PY{n}{step\PYZus{}11}\PY{o}{=}\PY{n}{simplify}\PY{p}{(}\PY{n}{psi\PYZus{}2}\PY{o}{.}\PY{n}{subs}\PY{p}{(}\PY{n}{x}\PY{p}{,}\PY{n}{a}\PY{p}{)}\PY{o}{\PYZhy{}}\PY{n}{psi\PYZus{}3}\PY{o}{.}\PY{n}{subs}\PY{p}{(}\PY{n}{x}\PY{p}{,}\PY{n}{a}\PY{p}{)}\PY{p}{)}
        \PY{n}{display}\PY{p}{(}\PY{l+s+s1}{\PYZsq{}}\PY{l+s+s1}{En x= a}\PY{l+s+s1}{\PYZsq{}}\PY{p}{,}\PY{n}{Eq}\PY{p}{(}\PY{n}{step\PYZus{}11}\PY{p}{,}\PY{l+m+mi}{0}\PY{p}{)}\PY{p}{)}
        \PY{n}{step\PYZus{}11d}\PY{o}{=}\PY{n}{simplify}\PY{p}{(}\PY{n}{diff}\PY{p}{(}\PY{n}{psi\PYZus{}2}\PY{p}{,}\PY{n}{x}\PY{p}{)}\PY{o}{.}\PY{n}{subs}\PY{p}{(}\PY{n}{x}\PY{p}{,}\PY{n}{a}\PY{p}{)}\PYZbs{}
                          \PY{o}{\PYZhy{}}\PY{n}{diff}\PY{p}{(}\PY{n}{psi\PYZus{}3}\PY{p}{,}\PY{n}{x}\PY{p}{)}\PY{o}{.}\PY{n}{subs}\PY{p}{(}\PY{n}{x}\PY{p}{,}\PY{n}{a}\PY{p}{)}\PY{p}{)}
        \PY{n}{display}\PY{p}{(}\PY{l+s+s1}{\PYZsq{}}\PY{l+s+s1}{En x= a}\PY{l+s+s1}{\PYZsq{}}\PY{p}{,}\PY{n}{Eq}\PY{p}{(}\PY{n}{step\PYZus{}11d}\PY{p}{,}\PY{l+m+mi}{0}\PY{p}{)}\PY{p}{)}
\end{Verbatim}

    
    \begin{verbatim}
'En x= -a'
    \end{verbatim}

    
    \[A e^{i a k} - B e^{- i a q} - C e^{i a q} + e^{- i a k} = 0\]

    
    
    \begin{verbatim}
'En x= -a'
    \end{verbatim}

    
    \[- i A k e^{i a k} - i B q e^{- i a q} + i C q e^{i a q} + i k e^{- i a k} = 0\]

    
    
    \begin{verbatim}
'En x= a'
    \end{verbatim}

    
    \[B e^{i a q} + C e^{- i a q} - D e^{i a k} = 0\]

    
    
    \begin{verbatim}
'En x= a'
    \end{verbatim}

    
    \[i B q e^{i a q} - i C q e^{- i a q} - i D k e^{i a k} = 0\]

    
    \begin{Verbatim}[commandchars=\\\{\}]
{\color{incolor}In [{\color{incolor}3}]:} \PY{n}{sols} \PY{o}{=}\PY{n}{solve}\PY{p}{(}\PY{p}{[}\PY{n}{step\PYZus{}1}\PY{p}{,} \PY{n}{step\PYZus{}1d}\PY{p}{,} \PY{n}{step\PYZus{}11}\PY{p}{,} \PY{n}{step\PYZus{}11d}\PY{p}{]}\PY{p}{)}\PY{p}{[}\PY{l+m+mi}{0}\PY{p}{]}
\end{Verbatim}

    \begin{Verbatim}[commandchars=\\\{\}]
{\color{incolor}In [{\color{incolor}4}]:} \PY{n}{R} \PY{o}{=} \PY{n}{trigsimp}\PY{p}{(}\PY{n}{expand}\PY{p}{(}\PY{n}{sols}\PY{p}{[}\PY{n}{A}\PY{p}{]}\PY{o}{*}\PY{n}{conjugate}\PY{p}{(}\PY{n}{sols}\PY{p}{[}\PY{n}{A}\PY{p}{]}\PY{p}{)}\PY{p}{)}\PYZbs{}
                     \PY{o}{.}\PY{n}{rewrite}\PY{p}{(}\PY{n}{sin}\PY{p}{)}\PY{p}{)}\PY{o}{.}\PY{n}{collect}\PY{p}{(}\PY{n}{cos}\PY{p}{(}\PY{l+m+mi}{4}\PY{o}{*}\PY{n}{a}\PY{o}{*}\PY{n}{q}\PY{p}{)}\PY{p}{)}
        \PY{n}{display}\PY{p}{(}\PY{n}{Eq}\PY{p}{(}\PY{n}{Symbol}\PY{p}{(}\PY{l+s+s1}{\PYZsq{}}\PY{l+s+s1}{R}\PY{l+s+s1}{\PYZsq{}}\PY{p}{)}\PY{p}{,}\PY{n}{R}\PY{p}{)}\PY{p}{)}
\end{Verbatim}

    \[R = \frac{\left(- k + q\right)^{2} \left(k + q\right)^{2} \left(\cos{\left (4 a q \right )} - 1\right)}{- k^{4} - 6 k^{2} q^{2} - q^{4} + \left(k^{4} - 2 k^{2} q^{2} + q^{4}\right) \cos{\left (4 a q \right )}}\]

    
    \[ R = \frac{(k^2-q^2)^2(\cos(4qa)-1)}{-(k^4+(8-2)k^2q^2+q^2)+(k^2-2k^2q^2+q^4)\cos(4qa)}\]

\[ R = \frac{(k^2-q^2)^2(\cos(4qa)-1)}{(k^4-2k^2q^2+q^4)(\cos(4qa)-1) -8k^2q^2}\]

\[ R = \frac{(k^2-q^2)^2(\cos(4qa)-1)}{(k^2-q^2)^2(\cos(4qa)-1) -8k^2q^2}\]

    \begin{Verbatim}[commandchars=\\\{\}]
{\color{incolor}In [{\color{incolor}5}]:} \PY{n}{T} \PY{o}{=} \PY{n}{trigsimp}\PY{p}{(}\PY{n}{expand}\PY{p}{(}\PY{p}{(}\PY{n}{sols}\PY{p}{[}\PY{n}{D}\PY{p}{]}\PY{o}{*}\PY{n}{conjugate}\PY{p}{(}\PY{n}{sols}\PY{p}{[}\PY{n}{D}\PY{p}{]}\PY{p}{)}\PY{p}{)}\PYZbs{}
                            \PY{o}{.}\PY{n}{rewrite}\PY{p}{(}\PY{n}{sin}\PY{p}{)}\PY{p}{)}\PY{p}{)}\PY{o}{.}\PY{n}{collect}\PY{p}{(}\PY{n}{cos}\PY{p}{(}\PY{l+m+mi}{4}\PY{o}{*}\PY{n}{a}\PY{o}{*}\PY{n}{q}\PY{p}{)}\PY{p}{)}
        \PY{n}{display}\PY{p}{(}\PY{n}{Eq}\PY{p}{(}\PY{n}{Symbol}\PY{p}{(}\PY{l+s+s1}{\PYZsq{}}\PY{l+s+s1}{T}\PY{l+s+s1}{\PYZsq{}}\PY{p}{)}\PY{p}{,}\PY{n}{T}\PY{p}{)}\PY{p}{)}
\end{Verbatim}

    \[T = \frac{8 k^{2} q^{2}}{k^{4} + 6 k^{2} q^{2} + q^{4} + \left(- k^{4} + 2 k^{2} q^{2} - q^{4}\right) \cos{\left (4 a q \right )}}\]

    
    \[ T=\frac{8k^2q^2}{(k^4 + (8-2)k^2q^2 + q^4)-(k^4-2k^2q^2+q^4)\cos(4aq)}\]
\[ T = \frac{8k^2q^2}{8k^2q^2+(k^2-q^2)^2(1-\cos(4aq))} \]

    \begin{Verbatim}[commandchars=\\\{\}]
{\color{incolor}In [{\color{incolor}6}]:} \PY{c+c1}{\PYZsh{}Comprobamos que se cumpla R+T=1}
        \PY{n}{simplify}\PY{p}{(}\PY{n}{R}\PY{o}{+}\PY{n}{T}\PY{p}{)}
\end{Verbatim}
\texttt{\color{outcolor}Out[{\color{outcolor}6}]:}
    
    \[1\]

    

    \begin{Verbatim}[commandchars=\\\{\}]
{\color{incolor}In [{\color{incolor}7}]:} \PY{c+c1}{\PYZsh{}Para ver que el coeficiente de transmisión se hace 1, lo graficamos.}
        \PY{c+c1}{\PYZsh{}Asumimos todas las constantes iguales a 1.}
        \PY{n}{E}\PY{p}{,} \PY{n}{m}\PY{p}{,} \PY{n}{h}\PY{p}{,} \PY{n}{V0} \PY{o}{=} \PY{n}{symbols}\PY{p}{(}\PY{l+s+s1}{\PYZsq{}}\PY{l+s+s1}{E m hbar V\PYZus{}0}\PY{l+s+s1}{\PYZsq{}}\PY{p}{,} \PY{n}{real}\PY{o}{=}\PY{n+nb+bp}{True}\PY{p}{,}\PYZbs{}
                              \PY{n}{positive}\PY{o}{=}\PY{n+nb+bp}{True}\PY{p}{)}
        \PY{n}{T\PYZus{}graph}\PY{o}{=}\PY{n}{T}\PY{o}{.}\PY{n}{subs}\PY{p}{(}\PY{n}{k}\PY{p}{,}\PY{n}{sqrt}\PY{p}{(}\PY{l+m+mi}{2}\PY{o}{*}\PY{n}{m}\PY{o}{*}\PY{n}{E}\PY{o}{/}\PY{p}{(}\PY{n}{h}\PY{o}{*}\PY{o}{*}\PY{l+m+mi}{2}\PY{p}{)}\PY{p}{)}\PY{p}{)}\PY{o}{.}\PYZbs{}
        \PY{n}{subs}\PY{p}{(}\PY{n}{q}\PY{p}{,} \PY{n}{sqrt}\PY{p}{(}\PY{l+m+mi}{2}\PY{o}{*}\PY{n}{m}\PY{o}{*}\PY{p}{(}\PY{n}{E}\PY{o}{\PYZhy{}}\PY{n}{V0}\PY{p}{)}\PY{o}{/}\PY{p}{(}\PY{n}{h}\PY{o}{*}\PY{o}{*}\PY{l+m+mi}{2}\PY{p}{)}\PY{p}{)}\PY{p}{)}\PYZbs{}
        \PY{o}{.}\PY{n}{subs}\PY{p}{(}\PY{n}{m}\PY{p}{,}\PY{l+m+mi}{1}\PY{p}{)}\PY{o}{.}\PY{n}{subs}\PY{p}{(}\PY{n}{h}\PY{p}{,}\PY{l+m+mi}{1}\PY{p}{)}\PY{o}{.}\PY{n}{subs}\PY{p}{(}\PY{n}{V0}\PY{p}{,}\PY{l+m+mi}{1}\PY{p}{)}\PY{o}{.}\PYZbs{}
        \PY{n}{subs}\PY{p}{(}\PY{n}{a}\PY{p}{,}\PY{l+m+mi}{1}\PY{p}{)}
        \PY{n}{R\PYZus{}graph}\PY{o}{=}\PY{n}{R}\PY{o}{.}\PY{n}{subs}\PY{p}{(}\PY{n}{k}\PY{p}{,}\PY{n}{sqrt}\PY{p}{(}\PY{l+m+mi}{2}\PY{o}{*}\PY{n}{m}\PY{o}{*}\PY{n}{E}\PY{o}{/}\PY{p}{(}\PY{n}{h}\PY{o}{*}\PY{o}{*}\PY{l+m+mi}{2}\PY{p}{)}\PY{p}{)}\PY{p}{)}\PYZbs{}
        \PY{o}{.}\PY{n}{subs}\PY{p}{(}\PY{n}{q}\PY{p}{,} \PY{n}{sqrt}\PY{p}{(}\PY{l+m+mi}{2}\PY{o}{*}\PY{n}{m}\PY{o}{*}\PY{p}{(}\PY{n}{E}\PY{o}{\PYZhy{}}\PY{n}{V0}\PY{p}{)}\PY{o}{/}\PY{p}{(}\PY{n}{h}\PY{o}{*}\PY{o}{*}\PY{l+m+mi}{2}\PY{p}{)}\PY{p}{)}\PY{p}{)}\PYZbs{}
        \PY{o}{.}\PY{n}{subs}\PY{p}{(}\PY{n}{m}\PY{p}{,}\PY{l+m+mi}{1}\PY{p}{)}\PY{o}{.}\PY{n}{subs}\PY{p}{(}\PY{n}{h}\PY{p}{,}\PY{l+m+mi}{1}\PY{p}{)}\PY{o}{.}\PY{n}{subs}\PY{p}{(}\PY{n}{V0}\PY{p}{,}\PY{l+m+mi}{1}\PY{p}{)}\PY{o}{.}\PY{n}{subs}\PY{p}{(}\PY{n}{a}\PY{p}{,}\PY{l+m+mi}{1}\PY{p}{)}
        \PY{n}{p1}\PY{o}{=}\PY{n}{plot}\PY{p}{(}\PY{n}{T\PYZus{}graph}\PY{p}{,} \PY{n}{label}\PY{o}{=}\PY{l+s+s1}{\PYZsq{}}\PY{l+s+s1}{T}\PY{l+s+s1}{\PYZsq{}}\PY{p}{,} \PY{n}{color}\PY{o}{=}\PY{l+s+s1}{\PYZsq{}}\PY{l+s+s1}{b}\PY{l+s+s1}{\PYZsq{}}\PY{p}{,}\PYZbs{}
                \PY{n}{ylabel}\PY{o}{=}\PY{l+s+s1}{\PYZsq{}}\PY{l+s+s1}{T}\PY{l+s+s1}{\PYZsq{}}\PY{p}{,} \PY{n}{xlim}\PY{o}{=}\PY{p}{(}\PY{l+m+mi}{0}\PY{p}{,}\PY{l+m+mi}{10}\PY{p}{)}\PY{p}{)}
        \PY{n}{display}\PY{p}{(}\PY{n}{solve}\PY{p}{(}\PY{n}{T}\PY{o}{.}\PY{n}{subs}\PY{p}{(}\PY{n}{k}\PY{p}{,}\PY{n}{sqrt}\PY{p}{(}\PY{l+m+mi}{2}\PY{o}{*}\PY{n}{m}\PY{o}{*}\PY{n}{E}\PY{o}{/}\PY{p}{(}\PY{n}{h}\PY{o}{*}\PY{o}{*}\PY{l+m+mi}{2}\PY{p}{)}\PY{p}{)}\PY{p}{)}\PY{o}{.}\PYZbs{}
                      \PY{n}{subs}\PY{p}{(}\PY{n}{q}\PY{p}{,} \PY{n}{sqrt}\PY{p}{(}\PY{l+m+mi}{2}\PY{o}{*}\PY{n}{m}\PY{o}{*}\PY{p}{(}\PY{n}{E}\PY{o}{\PYZhy{}}\PY{n}{V0}\PY{p}{)}\PY{o}{/}\PY{p}{(}\PY{n}{h}\PY{o}{*}\PY{o}{*}\PY{l+m+mi}{2}\PY{p}{)}\PY{p}{)}\PY{p}{)}\PY{p}{,}\PY{n}{E}\PY{p}{)}\PY{p}{)}
\end{Verbatim}

    \begin{center}
    \adjustimage{max size={0.9\linewidth}{0.9\paperheight}}{Sol_HW_3_files/Sol_HW_3_12_0.png}
    \end{center}
    { \hspace*{\fill} \\}
    
    \[\left [ \right ]\]

    
    \begin{Verbatim}[commandchars=\\\{\}]
{\color{incolor}In [{\color{incolor}8}]:} \PY{c+c1}{\PYZsh{}Hallamos para cuales energías se obtiene T=1.}
        \PY{n}{E\PYZus{}T1}\PY{o}{=}\PY{n}{solve}\PY{p}{(}\PY{n}{Eq}\PY{p}{(}\PY{n}{T}\PY{o}{.}\PY{n}{subs}\PY{p}{(}\PY{n}{k}\PY{p}{,}\PY{n}{sqrt}\PY{p}{(}\PY{l+m+mi}{2}\PY{o}{*}\PY{n}{m}\PY{o}{*}\PY{n}{E}\PY{o}{/}\PY{p}{(}\PY{n}{h}\PY{o}{*}\PY{o}{*}\PY{l+m+mi}{2}\PY{p}{)}\PY{p}{)}\PY{p}{)}\PY{o}{.}\PYZbs{}
                      \PY{n}{subs}\PY{p}{(}\PY{n}{q}\PY{p}{,} \PY{n}{sqrt}\PY{p}{(}\PY{l+m+mi}{2}\PY{o}{*}\PY{n}{m}\PY{o}{*}\PY{p}{(}\PY{n}{E}\PY{o}{\PYZhy{}}\PY{n}{V0}\PY{p}{)}\PY{o}{/}\PY{p}{(}\PY{n}{h}\PY{o}{*}\PY{o}{*}\PY{l+m+mi}{2}\PY{p}{)}\PY{p}{)}\PY{p}{)}\PY{p}{,}\PY{l+m+mi}{1}\PY{p}{)}\PY{p}{,}\PY{n}{E}\PY{p}{)}\PY{p}{[}\PY{l+m+mi}{0}\PY{p}{]}
        \PY{n}{display}\PY{p}{(}\PY{n}{Eq}\PY{p}{(}\PY{n}{E}\PY{p}{,}\PY{n}{E\PYZus{}T1}\PY{p}{)}\PY{p}{)}
\end{Verbatim}

    \[E = V_{0} + \frac{\pi^{2} \hbar^{2}}{8 a^{2} m}\]

    
    \begin{Verbatim}[commandchars=\\\{\}]
{\color{incolor}In [{\color{incolor}9}]:} \PY{c+c1}{\PYZsh{}La profundidad del potencial está dada por}
        \PY{k+kn}{import} \PY{n+nn}{scipy.constants} \PY{k+kn}{as} \PY{n+nn}{sc}
        \PY{n}{N}\PY{p}{(}\PY{n}{E\PYZus{}T1}\PY{o}{.}\PY{n}{subs}\PY{p}{(}\PY{n}{a}\PY{p}{,}\PY{l+m+mf}{0.1e\PYZhy{}9}\PY{p}{)}\PY{o}{.}\PY{n}{subs}\PY{p}{(}\PY{n}{h}\PY{o}{*}\PY{o}{*}\PY{l+m+mi}{2}\PY{p}{,}\PY{n}{sc}\PY{o}{.}\PY{n}{hbar}\PY{o}{*}\PY{o}{*}\PY{l+m+mi}{2}\PY{p}{)}\PYZbs{}
          \PY{o}{.}\PY{n}{subs}\PY{p}{(}\PY{n}{m}\PY{p}{,}\PY{n}{sc}\PY{o}{.}\PY{n}{m\PYZus{}e}\PY{p}{)}\PY{o}{\PYZhy{}}\PY{l+m+mf}{0.7}\PY{o}{*}\PY{n}{sc}\PY{o}{.}\PY{n}{e}\PY{p}{)} \PY{c+c1}{\PYZsh{}=0}
\end{Verbatim}
\texttt{\color{outcolor}Out[{\color{outcolor}9}]:}
    
    \[V_{0} + 1.39401446049131 \cdot 10^{-18}\]

    

    Entonces la profundidad del potencial es
\[ V_0 = -1.39\times10^{-18}J=-8.7\ eV\]

    \begin{Verbatim}[commandchars=\\\{\}]
{\color{incolor}In [{\color{incolor}10}]:} \PY{n}{p2}\PY{o}{=}\PY{n}{plot}\PY{p}{(}\PY{n}{R\PYZus{}graph}\PY{p}{,} \PY{n}{label}\PY{o}{=}\PY{l+s+s1}{\PYZsq{}}\PY{l+s+s1}{R}\PY{l+s+s1}{\PYZsq{}}\PY{p}{,} \PY{n}{color}\PY{o}{=}\PY{l+s+s1}{\PYZsq{}}\PY{l+s+s1}{r}\PY{l+s+s1}{\PYZsq{}}\PY{p}{,}\PYZbs{}
                 \PY{n}{ylabel}\PY{o}{=}\PY{l+s+s1}{\PYZsq{}}\PY{l+s+s1}{R}\PY{l+s+s1}{\PYZsq{}}\PY{p}{,} \PY{n}{xlim}\PY{o}{=}\PY{p}{(}\PY{l+m+mi}{0}\PY{p}{,}\PY{l+m+mi}{10}\PY{p}{)}\PY{p}{)}
\end{Verbatim}

    \begin{center}
    \adjustimage{max size={0.9\linewidth}{0.9\paperheight}}{Sol_HW_3_files/Sol_HW_3_16_0.png}
    \end{center}
    { \hspace*{\fill} \\}
    
    El fenómeno puede explicarse mirando la gráfica del coeficiente de
reflexión arriba, cuando lo átomos van muy despacio (baja energía) serán
siempre reflejados por la barrera de potencial (\(R=1\)).

    \begin{Verbatim}[commandchars=\\\{\}]
{\color{incolor}In [{\color{incolor} }]:} 
\end{Verbatim}


    % Add a bibliography block to the postdoc
    
    
    
    \end{document}
